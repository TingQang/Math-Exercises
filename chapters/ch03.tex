% 第3章:一元函数积分学
\chapter{一元函数积分学}
\label{ch:integral-calculus}

本章介绍不定积分和定积分的概念、性质、计算方法以及应用,包括牛顿-莱布尼茨公式、积分的应用等。

% 题型3.1 不定积分的计算
\begin{question-type}{不定积分的计算}
本题型主要考查基本积分公式、换元积分法、分部积分法、有理函数积分、三角函数积分等。

\begin{exercise}
计算以下不定积分:
\begin{enumerate}[label={【\arabic*】}]
    \item $\int x^2 \dx$ \difficulty{1} \exercisesource{基本积分公式}
    \item $\int \sin x \cos x \dx$ \difficulty{1} \exercisesource{李永乐复习全书P201}
    \item $\int \frac{1}{x^2 + 1} \dx$ \difficulty{2} \exercisesource{2024考研数学一真题}
    \item $\int x e^x \dx$ \difficulty{2} \exercisesource{B站 BV1xx4y1E7xx}
    \item $\int \frac{x}{x^2 + 1} \dx$ \difficulty{2} \exercisesource{有理函数积分}
\end{enumerate}
\end{exercise}

\begin{exercise}
使用换元法计算不定积分:
\begin{enumerate}[label={【\arabic*】}]
    \item $\int \frac{1}{\sqrt{1 - x^2}} \dx$ \difficulty{2} \exercisesource{三角换元}
    \item $\int x \sqrt{x + 1} \dx$ \difficulty{3} \exercisesource{李永乐复习全书P215}
    \item $\int \frac{\ln x}{x} \dx$ \difficulty{3} \exercisesource{对数函数积分}
\end{enumerate}
\end{exercise}

\begin{exercise}
使用分部积分法计算:
\begin{enumerate}[label={【\arabic*】}]
    \item $\int x \sin x \dx$ \difficulty{2} \exercisesource{分部积分基础}
    \item $\int \ln x \dx$ \difficulty{3} \exercisesource{2023考研数学一真题}
    \item $\int x^2 e^x \dx$ \difficulty{3} \exercisesource{高等数学教材}
\end{enumerate}
\end{exercise}
\end{question-type}

% 题型3.2 定积分的计算
\begin{question-type}{定积分的计算}
本题型主要考查定积分的定义、性质、牛顿-莱布尼茨公式、换元积分法、分部积分法在定积分中的应用。

\begin{exercise}
计算以下定积分:
\begin{enumerate}[label={【\arabic*】}]
    \item $\int_0^1 x^2 \dx$ \difficulty{1} \exercisesource{定积分基础}
    \item $\int_0^{\pi/2} \sin x \dx$ \difficulty{1} \exercisesource{李永乐复习全书P235}
    \item $\int_1^e \frac{1}{x} \dx$ \difficulty{2} \exercisesource{2024考研数学一真题}
    \item $\int_0^1 \frac{1}{\sqrt{1 + x^2}} \dx$ \difficulty{3} \exercisesource{B站 BV1xx4y1E7xx}
\end{enumerate}
\end{exercise}

\begin{exercise}
使用换元法计算定积分:
\begin{enumerate}[label={【\arabic*】}]
    \item $\int_0^1 x e^{x^2} \dx$ \difficulty{2} \exercisesource{换元积分}
    \item $\int_0^{\pi/4} \tan x \dx$ \difficulty{3} \exercisesource{三角函数积分}
    \item $\int_1^2 \frac{\ln x}{x} \dx$ \difficulty{4} \exercisesource{李永乐复习全书P248}
\end{enumerate}
\end{exercise}

\begin{exercise}
使用分部积分法计算定积分:
\begin{enumerate}[label={【\arabic*】}]
    \item $\int_0^1 x e^{-x} \dx$ \difficulty{3} \exercisesource{分部积分定积分}
    \item $\int_0^{\pi} x \sin x \dx$ \difficulty{3} \exercisesource{2023考研数学一真题}
\end{enumerate}
\end{exercise}
\end{question-type}

% 题型3.3 积分的应用
\begin{question-type}{积分的应用}
本题型主要考查定积分的几何意义(平面图形的面积、旋转体的体积)、物理意义(变力做功、质心等)。

\begin{exercise}
利用定积分计算平面图形的面积:
\begin{enumerate}[label={【\arabic*】}]
    \item 两条抛物线 $y = x^2$ 和 $y = 2 - x^2$ 所围成的图形的面积 \difficulty{2} \exercisesource{面积计算}
    \item 曲线 $y = \sin x$ 与 $x$ 轴在 $[0, \pi]$ 上围成的面积 \difficulty{2} \exercisesource{李永乐复习全书P265}
    \item 曲线 $y = e^x$、$y = e^{-x}$ 与直线 $x = 1$ 围成的面积 \difficulty{3} \exercisesource{2024考研数学一真题}
\end{enumerate}
\end{exercise}

\begin{exercise}
计算旋转体的体积:
\begin{enumerate}[label={【\arabic*】}]
    \item 曲线 $y = \sqrt{x}$ 在 $[0, 1]$ 上绕 $x$ 轴旋转所得旋转体的体积 \difficulty{3} \exercisesource{旋转体体积}
    \item 曲线 $y = \sin x$ 在 $[0, \pi]$ 上绕 $y$ 轴旋转所得旋转体的体积 \difficulty{4} \exercisesource{B站 BV1xx4y1E7xx}
\end{enumerate}
\end{exercise}

\begin{exercise}
积分在物理中的应用:
\begin{enumerate}[label={【\arabic*】}]
    \item 计算变力 $F(x) = x^2$ 将物体从 $x = 1$ 移动到 $x = 3$ 所做的功 \difficulty{2} \exercisesource{变力做功}
    \item 求密度为 $\rho(x) = x$ 的杆在 $[0, 1]$ 上的质心 \difficulty{3} \exercisesource{质心计算}
\end{enumerate}
\end{exercise}
\end{question-type} 
