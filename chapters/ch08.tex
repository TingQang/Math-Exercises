% 第8章:线性代数
\chapter{线性代数}
\label{ch:linear-algebra}

本章介绍行列式、矩阵、线性方程组、二次型等线性代数的基本概念、性质和计算方法。

% 题型8.1 行列式
\begin{question-type}{行列式}
本题型主要考查行列式的计算、性质以及应用。

\begin{exercise}
计算行列式:
\begin{enumerate}[label={【\arabic*】}]
    \item $\det \begin{pmatrix} 1 & 2 \\ 3 & 4 \end{pmatrix}$ \difficulty{1} \exercisesource{行列式基础}
    \item $\det \begin{pmatrix} 1 & 2 & 3 \\ 4 & 5 & 6 \\ 7 & 8 & 9 \end{pmatrix}$ \difficulty{2} \exercisesource{李永乐复习全书P701}
    \item $\det \begin{pmatrix} a & b & c \\ b & c & a \\ c & a & b \end{pmatrix}$ \difficulty{3} \exercisesource{2024考研数学一真题}
\end{enumerate}
\end{exercise}

\begin{exercise}
利用行列式性质计算:
\begin{enumerate}[label={【\arabic*】}]
    \item $\det \begin{pmatrix} 1 & 1 & 1 \\ 1 & 2 & 3 \\ 1 & 3 & 6 \end{pmatrix}$ \difficulty{2} \exercisesource{性质应用}
    \item $\det \begin{pmatrix} 1 & a & a^2 \\ 1 & b & b^2 \\ 1 & c & c^2 \end{pmatrix}$ \difficulty{3} \exercisesource{B站 BV1xx4y1E7xx}
\end{enumerate}
\end{exercise}

\begin{exercise}
抽象行列式的计算:
\begin{enumerate}[label={【\arabic*】}]
    \item 设 $A, B$ 为 n 阶矩阵,证明 $\det(AB) = \det A \cdot \det B$ \difficulty{3} \exercisesource{行列式性质}
    \item 计算 $\det(A + B)$ 的表达式 \difficulty{4} \exercisesource{李永乐复习全书P715}
\end{enumerate}
\end{exercise}
\end{question-type}

% 题型8.2 矩阵
\begin{question-type}{矩阵}
本题型主要考查矩阵的运算、逆矩阵、矩阵的特征值与特征向量、相似对角化等。

\begin{exercise}
矩阵的基本运算:
\begin{enumerate}[label={【\arabic*】}]
    \item 计算 $A + B$ 和 $AB$,其中 $A = \begin{pmatrix} 1 & 2 \\ 3 & 4 \end{pmatrix}, B = \begin{pmatrix} 5 & 6 \\ 7 & 8 \end{pmatrix}$ \difficulty{1} \exercisesource{矩阵运算}
    \item 计算 $A^2 - 3A + I$ \difficulty{2} \exercisesource{李永乐复习全书P735}
\end{enumerate}
\end{exercise}

\begin{exercise}
求逆矩阵:
\begin{enumerate}[label={【\arabic*】}]
    \item $A = \begin{pmatrix} 1 & 2 \\ 3 & 4 \end{pmatrix}$ 的逆矩阵 \difficulty{2} \exercisesource{逆矩阵计算}
    \item $A = \begin{pmatrix} 1 & 0 & 0 \\ 0 & 1 & 0 \\ 0 & 0 & 1 \end{pmatrix}$ 的逆矩阵 \difficulty{1} \exercisesource{B站 BV1xx4y1E7xx}
    \item 抽象矩阵 $A$ 的逆矩阵条件 \difficulty{3} \exercisesource{2024考研数学一真题}
\end{enumerate}
\end{exercise}

\begin{exercise}
矩阵的特征值与特征向量:
\begin{enumerate}[label={【\arabic*】}]
    \item $A = \begin{pmatrix} 2 & 1 \\ 1 & 2 \end{pmatrix}$ 的特征值和特征向量 \difficulty{3} \exercisesource{特征值计算}
    \item $A = \begin{pmatrix} 1 & 0 & 0 \\ 0 & 1 & 1 \\ 0 & 0 & 1 \end{pmatrix}$ 的特征值和特征向量 \difficulty{3} \exercisesource{李永乐复习全书P748}
\end{enumerate}
\end{exercise}

\begin{exercise}
矩阵的对角化:
\begin{enumerate}[label={【\arabic*】}]
    \item 判断矩阵是否可对角化 \difficulty{4} \exercisesource{对角化}
    \item 求矩阵的相似对角矩阵 \difficulty{4} \exercisesource{2023考研数学一真题}
\end{enumerate}
\end{exercise}
\end{question-type}

% 题型8.3 线性方程组
\begin{question-type}{线性方程组}
本题型主要考查线性方程组的解的判定、求解方法以及应用。

\begin{exercise}
解线性方程组:
\begin{enumerate}[label={【\arabic*】}]
    \item $\begin{cases} x + y = 1 \\ 2x + 3y = 4 \end{cases}$ \difficulty{1} \exercisesource{方程组基础}
    \item $\begin{cases} x + y + z = 1 \\ 2x + 3y + z = 2 \\ x + 2y + 2z = 3 \end{cases}$ \difficulty{2} \exercisesource{李永乐复习全书P765}
\end{enumerate}
\end{exercise}

\begin{exercise}
讨论线性方程组的解的情况:
\begin{enumerate}[label={【\arabic*】}]
    \item 当参数 $\lambda$ 取何值时,方程组有解 \difficulty{3} \exercisesource{解的判定}
    \item 齐次方程组的解的性质 \difficulty{3} \exercisesource{B站 BV1xx4y1E7xx}
\end{enumerate}
\end{exercise}

\begin{exercise}
用矩阵方法解方程组:
\begin{enumerate}[label={【\arabic*】}]
    \item $AX = B$ 的解 \difficulty{2} \exercisesource{矩阵方法}
    \item 克莱姆法则的应用 \difficulty{3} \exercisesource{2024考研数学一真题}
\end{enumerate}
\end{exercise}
\end{question-type}

% 题型8.4 二次型
\begin{question-type}{二次型}
本题型主要考查二次型的标准形、规范形、正定性判定等。

\begin{exercise}
化二次型为标准形:
\begin{enumerate}[label={【\arabic*】}]
    \item $f(x,y) = x^2 + 2xy + y^2$ \difficulty{2} \exercisesource{二次型标准形}
    \item $f(x,y,z) = x^2 + y^2 + z^2 + 2xy + 2xz + 2yz$ \difficulty{3} \exercisesource{李永乐复习全书P785}
\end{enumerate}
\end{exercise}

\begin{exercise}
判定二次型的正定性:
\begin{enumerate}[label={【\arabic*】}]
    \item $f(x,y) = x^2 + 2y^2$ \difficulty{2} \exercisesource{正定性判定}
    \item $f(x,y,z) = x^2 + y^2 + z^2 + 2xy + 2xz + 2yz$ \difficulty{4} \exercisesource{B站 BV1xx4y1E7xx}
\end{enumerate}
\end{exercise}

\begin{exercise}
用矩阵方法处理二次型:
\begin{enumerate}[label={【\arabic*】}]
    \item 二次型的矩阵表示 \difficulty{3} \exercisesource{矩阵方法}
    \item 合同变换与标准形 \difficulty{4} \exercisesource{2023考研数学一真题}
\end{enumerate}
\end{exercise}
\end{question-type} 
