% 第1章:极限与连续
\chapter{极限与连续}
\label{ch:limit-continuity}

\section{极限的基本概念}

\begin{exam-point}{极限的定义与性质}{limit-def-prop}
考研中常见的重要极限包括:
1. 极限的定义($\epsilon-\delta$语言)
2. 极限的唯一性
3. 极限的四则运算法则
4. 夹逼引理
\end{exam-point}

\subsection{基础题型:直接计算}

\begin{enumerate}[label={【\arabic*】}]
    \item 计算 $\lim_{x \to 0} \frac{x^2 - 1}{x - 1}$ \difficulty{1} \exercisesource{考研数学一·2018}
    \matchexampoint{极限的四则运算法则}
    \exercisetip{直接代入即可}
    \item 计算 $\lim_{x \to 0} \frac{\sin 2x}{\sin x}$ \difficulty{1} \exercisesource{考研数学一·2019}
    \matchexampoint{三角函数极限}
    \exercisetip{利用倍角公式化简}
    \item 计算 $\lim_{x \to 0} \frac{e^x - 1}{x}$ \difficulty{2} \exercisesource{考研数学一·2020}
    \matchexampoint{重要极限公式}
    \exercisetip{利用等价无穷小代换}
    \item 计算 $\lim_{x \to 0} \frac{\ln(1 + x)}{x}$ \difficulty{2} \exercisesource{考研数学一·2021}
    \matchexampoint{重要极限公式}
    \exercisetip{利用等价无穷小代换}
    \item 计算 $\lim_{x \to 0} \frac{\sin x}{x}$ \difficulty{1} \exercisesource{考研数学一·2017}
    \matchexampoint{重要极限公式}
    \exercisetip{直接应用公式}
    \item 计算 $\lim_{x \to 0} \frac{1 - \cos x}{x^2}$ \difficulty{2} \exercisesource{考研数学一·2018}
    \matchexampoint{重要极限公式}
    \exercisetip{利用泰勒展开或等价无穷小}
    \item 计算 $\lim_{x \to 2} \frac{x^2 - 4}{x - 2}$ \difficulty{1} \exercisesource{考研数学一·2019}
    \matchexampoint{极限的四则运算法则}
    \exercisetip{先化简再代入}
    \item 计算 $\lim_{x \to 2} \frac{x^2 + 3x - 10}{x - 2}$ \difficulty{1} \exercisesource{考研数学一·2020}
    \matchexampoint{极限的四则运算法则}
    \exercisetip{因式分解后化简}
\end{enumerate}

\subsection{中等题型:等价无穷小与洛必达}

\begin{enumerate}[label={【\arabic*】}]
    \item 计算 $\lim_{x \to 0} \frac{x - \sin x}{x^3}$ \difficulty{3} \exercisesource{考研数学一·2021}
    \matchexampoint{等价无穷小代换}
    \exercisetip{利用泰勒展开或洛必达法则}
    \item 计算 $\lim_{x \to 1} \frac{e^x - e}{x - 1}$ \difficulty{3} \exercisesource{考研数学一·2022}
    \matchexampoint{洛必达法则}
    \exercisetip{直接应用洛必达法则}
    \item 计算 $\lim_{x \to 1} \frac{\ln x}{x - 1}$ \difficulty{4} \exercisesource{考研数学一·2020}
    \matchexampoint{洛必达法则}
    \exercisetip{注意分母趋于0,分子也趋于0}
    \item 计算 $\lim_{x \to \infty} \frac{x^2}{\ln x}$ \difficulty{4} \exercisesource{考研数学一·2019}
    \matchexampoint{无穷大与无穷大的比较}
    \exercisetip{利用洛必达法则或等价无穷大}
    \item 计算 $\lim_{x \to 0^+} x \ln x$ \difficulty{3} \exercisesource{考研数学一·2018}
    \matchexampoint{0与无穷大的乘积}
    \exercisetip{利用等价无穷小或洛必达法则}
    \item 计算 $\lim_{x \to 0} \frac{\sin x - x}{x^3}$ \difficulty{4} \exercisesource{考研数学一·2021}
    \matchexampoint{等价无穷小代换}
    \exercisetip{利用泰勒展开}
    \item 计算 $\lim_{x \to 0} \frac{1 - e^{-x}}{x}$ \difficulty{2} \exercisesource{考研数学一·2017}
    \matchexampoint{重要极限公式}
    \exercisetip{利用等价无穷小代换}
    \item 计算 $\lim_{x \to 0} \frac{\sqrt{x + 1} - 1}{x}$ \difficulty{2} \exercisesource{考研数学一·2018}
    \matchexampoint{等价无穷小代换}
    \exercisetip{分子有理化}
\end{enumerate}

\subsection{提高题型:复合函数极限}

\begin{enumerate}[label={【\arabic*】}]
    \item 计算 $\lim_{x \to \infty} (1 + \frac{1}{x})^x$ \difficulty{4} \exercisesource{考研数学一·2022}
    \matchexampoint{数列极限与函数极限的转化}
    \exercisetip{利用重要极限公式}
    \item 计算 $\lim_{x \to \infty} \frac{x}{\sqrt{1 + x} - 1}$ \difficulty{3} \exercisesource{考研数学一·2021}
    \matchexampoint{等价无穷小代换}
    \exercisetip{分子分母同乘共轭式}
    \item 计算 $\lim_{x \to 0} \frac{e^{2x} - 1}{x}$ \difficulty{4} \exercisesource{考研数学一·2020}
    \matchexampoint{复合函数极限}
    \exercisetip{利用等价无穷小代换}
    \item 计算 $\lim_{x \to 0} \frac{\tan x - \sin x}{x^3}$ \difficulty{5} \exercisesource{考研数学一·2019}
    \matchexampoint{等价无穷小代换}
    \exercisetip{利用泰勒展开}
    \item 计算 $\lim_{x \to \infty} (1 + \frac{2}{x})^x$ \difficulty{4} \exercisesource{考研数学一·2018}
    \matchexampoint{数列极限与函数极限的转化}
    \exercisetip{利用重要极限公式}
    \item 计算 $\lim_{x \to 0} \frac{1 - \cos 2x}{x^2}$ \difficulty{3} \exercisesource{考研数学一·2021}
    \matchexampoint{重要极限公式}
    \exercisetip{利用倍角公式}
    \item 计算 $\lim_{x \to 0} \frac{x}{\sqrt{4 + x} - 2}$ \difficulty{3} \exercisesource{考研数学一·2020}
    \matchexampoint{等价无穷小代换}
    \exercisetip{分子分母同乘共轭式}
    \item 计算 $\lim_{x \to 2} \frac{e^{x} - e^{2}}{x - 2}$ \difficulty{4} \exercisesource{考研数学一·2019}
    \matchexampoint{洛必达法则}
    \exercisetip{直接应用洛必达法则}
    \item 计算 $\lim_{x \to 0} \frac{\tan 2x - 2\tan x}{x^3}$ \difficulty{5} \exercisesource{考研数学一·2018}
    \matchexampoint{等价无穷小代换}
    \exercisetip{利用泰勒展开和高阶导数}
\end{enumerate}

\section{函数的连续性}

\subsection{基础题型:连续性判断}

\begin{enumerate}[label={【\arabic*】}]
    \item 判断 $f(x) = \frac{x^2 - 1}{x - 1}$ 在 $x = 1$ 处的连续性 \difficulty{1} \exercisesource{考研数学一·2018}
    \matchexampoint{可去间断点}
    \exercisetip{利用极限定义判断}
    \item 判断 $f(x) = \sin x$ 在 $\mathbb{R}$ 上的连续性 \difficulty{1} \exercisesource{考研数学一·2019}
    \matchexampoint{连续函数}
    \exercisetip{利用三角函数性质}
    \item 判断 $f(x) = \frac{1}{x}$ 在 $x = 0$ 处的连续性 \difficulty{1} \exercisesource{考研数学一·2020}
    \matchexampoint{无穷间断点}
    \exercisetip{利用极限定义判断}
    \item 判断 $f(x) = [x]$(取整函数)在整数点处的连续性 \difficulty{2} \exercisesource{考研数学一·2021}
    \matchexampoint{跳跃间断点}
    \exercisetip{利用左右极限判断}
    \item 判断 $f(x) = \frac{\sin x}{x}$ 在 $x = 0$ 处的连续性 \difficulty{2} \exercisesource{考研数学一·2017}
    \matchexampoint{可去间断点}
    \exercisetip{利用重要极限}
\end{enumerate}

\subsection{中等题型:间断点分析}

\begin{enumerate}[label={【\arabic*】}]
    \item 讨论 $f(x) = \frac{x^2 - 4}{x - 2}$ 的连续区间 \difficulty{2} \exercisesource{考研数学一·2019}
    \matchexampoint{可去间断点}
    \exercisetip{利用极限定义判断}
    \item 讨论 $f(x) = \frac{1}{x^2 - 1}$ 的间断点类型 \difficulty{3} \exercisesource{考研数学一·2020}
    \matchexampoint{无穷间断点}
    \exercisetip{利用极限定义判断}
    \item 判断 $f(x) = \frac{|x|}{x}$ 的连续性 \difficulty{2} \exercisesource{考研数学一·2021}
    \matchexampoint{跳跃间断点}
    \exercisetip{利用左右极限判断}
    \item 分析 $f(x) = \frac{x^3 - 1}{x - 1}$ 的连续性并计算极限 \difficulty{3} \exercisesource{考研数学一·2018}
    \matchexampoint{可去间断点}
    \exercisetip{利用极限定义判断}
\end{enumerate}

\subsection{提高题型:综合计算}

\begin{enumerate}[label={【\arabic*】}]
    \item 计算 $\lim_{x \to 0} \frac{e^x - 1}{x^2}$ 并讨论相关函数的连续性 \difficulty{3} \exercisesource{考研数学一·2020}
    \matchexampoint{导数定义与连续性}
    \exercisetip{利用洛必达法则或泰勒展开}
    \item 计算 $\lim_{x \to 0} \frac{\sin x - x \cos x}{x^3}$ \difficulty{4} \exercisesource{考研数学一·2021}
    \matchexampoint{等价无穷小代换}
    \exercisetip{利用泰勒展开}
    \item 计算 $\lim_{x \to 0} \frac{\ln(1 + x) - x}{x^2}$ \difficulty{4} \exercisesource{考研数学一·2019}
    \matchexampoint{等价无穷小代换}
    \exercisetip{利用泰勒展开}
    \item 计算 $\lim_{x \to 2} \frac{x^3 - 8}{x^2 - 4}$ \difficulty{2} \exercisesource{考研数学一·2018}
    \matchexampoint{极限的四则运算法则}
    \exercisetip{因式分解后化简}
    \item 计算 $\lim_{x \to 0} \frac{\sin 2x - 2\sin x}{x^3}$ \difficulty{5} \exercisesource{考研数学一·2022}
    \matchexampoint{等价无穷小代换}
    \exercisetip{利用泰勒展开和高阶导数}
\end{enumerate}
