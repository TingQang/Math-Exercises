% 第9章:概率论基础
\chapter{概率论基础}
\label{ch:probability-basics}

本章介绍概率论的基本概念,包括随机事件、概率计算、随机变量及其分布、多维随机变量等。

% 题型9.1 随机事件与概率
\begin{question-type}{随机事件与概率}
本题型主要考查随机事件的运算、概率的基本性质、条件概率、全概率公式、贝叶斯公式等。

\begin{exercise}
随机事件的运算:
\begin{enumerate}[label={【\arabic*】}]
    \item 设 $A, B$ 为随机事件,证明 $P(A \cup B) = P(A) + P(B) - P(AB)$ \difficulty{2} \exercisesource{事件运算}
    \item 计算 $P((A \cup B)^c)$ \difficulty{2} \exercisesource{李永乐复习全书P801}
\end{enumerate}
\end{exercise}

\begin{exercise}
概率计算:
\begin{enumerate}[label={【\arabic*】}]
    \item 古典概型:从 52 张扑克牌中抽取一张是红桃的概率 \difficulty{1} \exercisesource{古典概型}
    \item 几何概型:随机投点到正方形,落在圆内的概率 \difficulty{2} \exercisesource{B站 BV1xx4y1E7xx}
    \item 伯努利概型:10 次独立试验,每次成功概率 0.3,恰好成功 3 次的概率 \difficulty{3} \exercisesource{2024考研数学一真题}
\end{enumerate}
\end{exercise}

\begin{exercise}
条件概率与全概率公式:
\begin{enumerate}[label={【\arabic*】}]
    \item 条件概率 $P(A|B)$ 的计算 \difficulty{2} \exercisesource{条件概率}
    \item 全概率公式应用:产品合格率问题 \difficulty{3} \exercisesource{李永乐复习全书P815}
    \item 贝叶斯公式:疾病诊断概率 \difficulty{3} \exercisesource{2023考研数学一真题}
\end{enumerate}
\end{exercise}
\end{question-type}

% 题型9.2 随机变量及其分布
\begin{question-type}{随机变量及其分布}
本题型主要考查离散型随机变量、连续型随机变量的分布函数、概率密度函数、常见分布等。

\begin{exercise}
离散型随机变量:
\begin{enumerate}[label={【\arabic*】}]
    \item 0-1 分布的参数与性质 \difficulty{1} \exercisesource{离散分布}
    \item 二项分布 $B(n,p)$ 的概率计算 \difficulty{2} \exercisesource{李永乐复习全书P835}
    \item 泊松分布的参数估计 \difficulty{3} \exercisesource{B站 BV1xx4y1E7xx}
\end{enumerate}
\end{exercise}

\begin{exercise}
连续型随机变量:
\begin{enumerate}[label={【\arabic*】}]
    \item 均匀分布 $U(a,b)$ 的密度函数 \difficulty{1} \exercisesource{连续分布}
    \item 正态分布 $N(\mu,\sigma^2)$ 的性质 \difficulty{2} \exercisesource{2024考研数学一真题}
    \item 指数分布的参数意义 \difficulty{3} \exercisesource{李永乐复习全书P848}
\end{enumerate}
\end{exercise}

\begin{exercise}
分布函数与密度函数:
\begin{enumerate}[label={【\arabic*】}]
    \item 求分布函数 $F(x)$ \difficulty{2} \exercisesource{分布函数}
    \item 由密度函数求概率 $P(a < X \leq b)$ \difficulty{2} \exercisesource{密度函数应用}
\end{enumerate}
\end{exercise}
\end{question-type}

% 题型9.3 多维随机变量
\begin{question-type}{多维随机变量}
本题型主要考查二维随机变量的联合分布、边缘分布、条件分布、独立性等。

\begin{exercise}
二维离散随机变量:
\begin{enumerate}[label={【\arabic*】}]
    \item 联合概率分布表 \difficulty{2} \exercisesource{二维离散}
    \item 边缘分布的计算 \difficulty{2} \exercisesource{李永乐复习全书P865}
\end{enumerate}
\end{exercise}

\begin{exercise}
二维连续随机变量:
\begin{enumerate}[label={【\arabic*】}]
    \item 联合密度函数的性质 \difficulty{3} \exercisesource{二维连续}
    \item 边缘密度函数的计算 \difficulty{3} \exercisesource{B站 BV1xx4y1E7xx}
\end{enumerate}
\end{exercise}

\begin{exercise}
随机变量的独立性:
\begin{enumerate}[label={【\arabic*】}]
    \item 判断两个随机变量是否独立 \difficulty{3} \exercisesource{独立性判定}
    \item 独立随机变量的性质 \difficulty{3} \exercisesource{2024考研数学一真题}
\end{enumerate}
\end{exercise}
\end{question-type}