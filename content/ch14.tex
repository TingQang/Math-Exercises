\chapter{随机事件和概率}

\section{考点说明}
\begin{itemize}
\item 随机试验、样本空间、随机事件;
\item 概率的公理化定义;
\item 条件概率;
\item 独立事件的性质;
\item 事件间的关系与运算;
\item 伯努利概型
\item 概率的基本性质;
\item 乘法公式、加法公式、全概率公式、贝叶斯公式
\item 事件独立性的概念;
\item 事件的运算律
\item 古典概型与几何概型;
\end{itemize}

\section{第一节 随机事件及其运算}

\pt{事件间关系的判断与运算;}

\pt{利用事件运算律化简事件表达式;}

\pt{事件的互斥、对立关系应用;}

\pt{复杂事件的表达式构造;}

\pt{事件运算律的证明题}

\section{第二节 概率的定义与性质}

\pt{古典概型概率计算;}

\pt{几何概型概率计算;}

\pt{条件概率的计算;}

\pt{加法公式、乘法公式的应用;}

\pt{全概率公式与贝叶斯公式的应用;}

\pt{古典概型中的排列组合综合应用;}

\pt{几何概型中多维空间的概率计算;}

\pt{全概率公式的多阶段问题应用;}

\pt{贝叶斯公式在决策问题中的应用;}

\pt{概率的不等式应用(如单调性、有界性)}

\section{第三节 事件的独立性}

\pt{事件独立性的判断;}

\pt{独立事件概率的计算;}

\pt{伯努利概型的概率计算;}

\pt{多个事件独立性的判断;}

\pt{独立重复试验的综合问题;}

\pt{伯努利概型中最可能成功次数的求解;}

\pt{独立性与互斥性的关系辨析}

