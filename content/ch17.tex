\chapter{随机变量的数字特征}

\section{考点说明}
\begin{itemize}
\item 数学期望的性质;
\item 标准差;
\item 相关系数的概念与性质;
\item 数学期望的概念;
\item 矩、协方差矩阵
\item 常见分布的方差;
\item 随机变量函数的数学期望;
\item 常见分布的数学期望
\item 协方差的概念与性质;
\item 方差的概念与性质;
\end{itemize}

\section{第一节 数学期望}

\pt{数学期望的计算(离散型、连续型);}

\pt{数学期望性质的应用;}

\pt{随机变量函数的数学期望计算;}

\pt{常见分布数学期望的应用;}

\pt{二维随机变量的数学期望计算;}

\pt{数学期望的不等式应用(如柯西-施瓦茨不等式);}

\pt{含参数随机变量的数学期望求解;}

\pt{数学期望在实际问题中的应用(如收益期望)}

\section{第二节 方差与协方差}

\pt{方差的计算;}

\pt{协方差与相关系数的计算;}

\pt{相关系数的性质应用;}

\pt{常见分布方差的应用;}

\pt{二维随机变量的方差、协方差计算;}

\pt{方差的性质综合应用(如独立变量方差可加性);}

\pt{相关系数与独立性的关系辨析;}

\pt{含参数随机变量的方差、协方差求解;}

\pt{协方差矩阵的计算与应用;}

\pt{原点矩与中心矩的计算;}

\pt{利用方差进行风险评估的应用问题}

\pt{利用方差进行风险评估的应用问题}

