\chapter{一元函数积分学}

\section{考点说明}
\begin{itemize}
\item 积分上限的函数及其导数;
\item 不定积分和定积分的换元积分法与分部积分法;
\item 基本积分公式;
\item 反常(广义)积分;
\item 定积分的概念和基本性质;
\item 定积分中值定理
\item 有理函数、三角函数的有理式和简单无理函数的积分
\item 反常积分收敛的比较判别法;
\item 不定积分的基本性质;
\item 函数的平均值
\item 定积分的应用(平面图形的面积、平面曲线的弧长、旋转体的体积及侧面积、平行截面面积为已知的立体体积、功、引力、压力、质心、形心等);
\item 原函数和不定积分的概念;
\item 牛顿-莱布尼茨(Newton-Leibniz)公式;
\end{itemize}

\section{第一节 不定积分与定积分的概念及性质}

\pt{原函数存在性判断;}

\pt{不定积分的基本计算;}

\pt{定积分的概念应用;}

\pt{定积分性质的应用;}

\pt{定积分中值定理的应用;}

\pt{利用定积分性质比较积分大小;}

\pt{原函数的构造与判断;}

\pt{含抽象函数的定积分性质应用}

\section{第二节 积分的计算方法}

\pt{积分上限函数的导数计算;}

\pt{牛顿-莱布尼茨公式的应用;}

\pt{换元积分法与分部积分法计算积分;}

\pt{有理函数积分;}

\pt{三角函数有理式积分;}

\pt{简单无理函数积分;}

\pt{含变限积分的方程求解;}

\pt{分部积分法的综合应用(多次分部、递推公式);}

\pt{利用对称性简化定积分计算;}

\pt{含绝对值、分段函数的定积分计算}

\section{第三节 反常积分与定积分的应用}

\pt{反常积分的计算与收敛性判断;}

\pt{平面图形面积的计算;}

\pt{平面曲线弧长的计算;}

\pt{旋转体体积与侧面积的计算;}

\pt{平行截面面积为已知的立体体积计算;}

\pt{定积分在物理中的应用;}

\pt{函数平均值的计算;}

\pt{反常积分敛散性的比较判别法应用;}

\pt{旋转体体积的综合计算(含挖去部分、组合旋转体);}

\pt{定积分在几何中的综合应用(面积与体积结合);}

\pt{物理应用中的变力做功、压力计算的综合问题}

