\chapter{向量}

\section{考点说明}
\begin{itemize}
\item 规范正交基;
\item 向量的线性组合与线性表示
\item 向量的概念;
\item 线性相关性的判定定理;
\item 基变换与坐标变换;
\item 向量的线性运算;
\item 基、维数与坐标;
\item 向量空间的概念;
\item 向量的内积与正交性
\item 向量组线性相关与线性无关的概念;
\item 向量组的秩与极大线性无关组
\end{itemize}

\section{第一节 向量的概念与线性运算}

\pt{向量的线性运算;}

\pt{向量的线性组合与线性表示判断;}

\pt{求解线性表示的系数;}

\pt{向量组的线性表示与矩阵秩的关系应用;}

\pt{含参数向量的线性表示讨论;}

\pt{多个向量线性表示的综合问题;}

\pt{向量线性表示的证明题}

\section{第二节 向量组的线性相关性}

\pt{向量组线性相关性的判断;}

\pt{向量组秩的计算与极大线性无关组的求解;}

\pt{线性相关性判定定理的应用;}

\pt{含参数向量组的线性相关性讨论;}

\pt{向量组秩的不等式应用;}

\pt{极大线性无关组的扩充与应用;}

\pt{线性相关性的证明题(如证明线性无关向量组的变形仍无关);}

\pt{向量组等价与秩的关系应用}

\section{第三节 向量空间}

\pt{向量空间基与维数的求解;}

\pt{向量在基下的坐标计算;}

\pt{基变换与坐标变换;}

\pt{规范正交基的构造(施密特正交化);}

\pt{过渡矩阵的计算与应用;}

\pt{子空间的基与维数求解;}

\pt{施密特正交化的综合应用;}

\pt{向量空间同构的判断;}

\pt{向量内积的计算与正交性判断}

