\chapter{函数、极限、连续}

\section{考点说明}
\begin{itemize}
\item 基本初等函数的性质及其图形;
\item 函数间断点的类型;
\item 极限存在的两个准则(单调有界准则和夹逼准则);
\item 函数连续的概念(含左连续与右连续);
\item 闭区间上连续函数的性质(有界性、最大值和最小值定理、介值定理)
\item 初等函数;
\item 函数关系的建立
\item 无穷小量和无穷大量的概念及其关系;
\item 两个重要极限:`limₓ→₀ sinx/x=1`、`limₓ→∞(1+1/x)ˣ=e`
\item 复合函数、反函数、分段函数和隐函数;
\item 初等函数的连续性;
\item 函数的概念及表示法;
\item 函数的左极限和右极限;
\item 函数的有界性、单调性、周期性和奇偶性;
\item 数列极限与函数极限的定义及其性质;
\item 无穷小量的性质及无穷小量的比较;
\item 极限的四则运算;
\end{itemize}

\section{第一节 函数}

\pt{复合函数解析式求解;}

\pt{函数奇偶性、周期性、有界性判定;}

\pt{分段函数定义域与值域求解;}

\pt{函数性质的综合应用;}

\pt{应用问题的函数关系建立;}

\pt{抽象函数的性质分析;}

\pt{函数图像的识别与绘制(结合初等函数性质)}

\section{第二节 极限}

\pt{极限的概念性问题(存在性判断、左右极限求解);}

\pt{利用四则运算、重要极限求极限;}

\pt{利用夹逼准则、单调有界准则求极限;}

\pt{无穷小量阶的比较;}

\pt{极限式中参数的确定;}

\pt{无穷大量与无界变量的区分;}

\pt{含变限积分的极限计算;}

\pt{利用泰勒公式求极限;}

\pt{数列极限的转化(转化为函数极限求解);}

\pt{含抽象函数的极限问题}

\section{第三节 连续}

\pt{函数连续性的判定;}

\pt{函数间断点的识别与类型判断;}

\pt{闭区间上连续函数性质的应用;}

\pt{利用连续性求极限;}

\pt{分段函数连续性的讨论(含参数);}

\pt{利用介值定理证明方程根的存在性}

