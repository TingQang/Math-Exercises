% 第1章:极限与连续
\chapter{极限与连续}
\label{ch:limit-continuity}

\section{极限的基本概念}

\begin{exam-point}{极限的定义与性质}{limit-def-prop}
考研中常见的重要极限包括:
1. 极限的定义($\epsilon-\delta$语言)
2. 极限的唯一性
3. 极限的四则运算法则
4. 夹逼引理
\end{exam-point}

\subsection{题型1:极限的四则运算法则}

\begin{enumerate}[label={【\arabic*】}]
    \qitem{ex:ch01-01-01} \difficulty{1} 计算 $\lim_{x \to 0} \frac{x^2 - 1}{x - 1}$ \exercisesource{考研数学一·2018}
    \matchexampoint{极限的四则运算法则}
    \exercisetip{直接代入即可}
    \qitem{ex:ch01-01-02} \difficulty{1} 计算 $\lim_{x \to 2} \frac{x^2 - 4}{x - 2}$ \exercisesource{考研数学一·2019}
    \matchexampoint{极限的四则运算法则}
    \exercisetip{先化简再代入}
    \qitem{ex:ch01-01-03} \difficulty{1} 计算 $\lim_{x \to 2} \frac{x^2 + 3x - 10}{x - 2}$ \exercisesource{考研数学一·2020}
    \matchexampoint{极限的四则运算法则}
    \exercisetip{因式分解后化简}
\end{enumerate}

\subsection{题型2:等价无穷小替换}

\begin{enumerate}[label={【\arabic*】}]
    \qitem{ex:ch01-02-01} \difficulty{1} 计算 $\lim_{x \to 0} \frac{\sin 2x}{\sin x}$ \exercisesource{考研数学一·2019}
    \matchexampoint{三角函数极限}
    \exercisetip{利用倍角公式化简}
    \qitem{ex:ch01-02-02} \difficulty{2} 计算 $\lim_{x \to 0} \frac{e^x - 1}{x}$ \exercisesource{考研数学一·2020}
    \matchexampoint{重要极限公式}
    \exercisetip{利用等价无穷小代换}
    \qitem{ex:ch01-02-03} \difficulty{2} 计算 $\lim_{x \to 0} \frac{\ln(1 + x)}{x}$ \exercisesource{考研数学一·2021}
    \matchexampoint{重要极限公式}
    \exercisetip{利用等价无穷小代换}
\end{enumerate}

\subsection{题型3:洛必达法则}

\begin{enumerate}[label={【\arabic*】}]
    \qitem{ex:ch01-03-01} \difficulty{3} 计算 $\lim_{x \to 1} \frac{e^x - e}{x - 1}$ \exercisesource{考研数学一·2022}
    \matchexampoint{洛必达法则}
    \exercisetip{直接应用洛必达法则}
    \qitem{ex:ch01-03-02} \difficulty{4} 计算 $\lim_{x \to 1} \frac{\ln x}{x - 1}$ \exercisesource{考研数学一·2020}
    \matchexampoint{洛必达法则}
    \exercisetip{注意分母趋于0,分子也趋于0}
    \qitem{ex:ch01-03-03} \difficulty{4} 计算 $\lim_{x \to \infty} \frac{x^2}{\ln x}$ \exercisesource{考研数学一·2019}
    \matchexampoint{无穷大与无穷大的比较}
    \exercisetip{利用洛必达法则或等价无穷大}
\end{enumerate}

\subsection{题型4:泰勒公式}

\begin{enumerate}[label={【\arabic*】}]
    \qitem{ex:ch01-04-01} \difficulty{3} 计算 $\lim_{x \to 0} \frac{x - \sin x}{x^3}$ \exercisesource{考研数学一·2021}
    \matchexampoint{等价无穷小代换}
    \exercisetip{利用泰勒展开或洛必达法则}
    \qitem{ex:ch01-04-02} \difficulty{4} 计算 $\lim_{x \to 0} \frac{\sin x - x}{x^3}$ \exercisesource{考研数学一·2021}
    \matchexampoint{等价无穷小代换}
    \exercisetip{利用泰勒展开}
    \qitem{ex:ch01-04-03} \difficulty{5} 计算 $\lim_{x \to 0} \frac{\tan x - \sin x}{x^3}$ \exercisesource{考研数学一·2019}
    \matchexampoint{等价无穷小代换}
    \exercisetip{利用泰勒展开}
\end{enumerate}

\subsection{题型5:重要极限公式}

\begin{enumerate}[label={【\arabic*】}]
    \qitem{ex:ch01-05-01} \difficulty{1} 计算 $\lim_{x \to 0} \frac{\sin x}{x}$ \exercisesource{考研数学一·2017}
    \matchexampoint{重要极限公式}
    \exercisetip{直接应用公式}
    \qitem{ex:ch01-05-02} \difficulty{4} 计算 $\lim_{x \to \infty} (1 + \frac{1}{x})^x$ \exercisesource{考研数学一·2022}
    \matchexampoint{数列极限与函数极限的转化}
    \exercisetip{利用重要极限公式}
    \qitem{ex:ch01-05-03} \difficulty{4} 计算 $\lim_{x \to \infty} (1 + \frac{2}{x})^x$ \exercisesource{考研数学一·2018}
    \matchexampoint{数列极限与函数极限的转化}
    \exercisetip{利用重要极限公式}
\end{enumerate}

\subsection{题型6:无穷大的比较}

\begin{enumerate}[label={【\arabic*】}]
    \qitem{ex:ch01-06-01} \difficulty{4} 计算 $\lim_{x \to \infty} \frac{x^2}{\ln x}$ \exercisesource{考研数学一·2019}
    \matchexampoint{无穷大与无穷大的比较}
    \exercisetip{利用洛必达法则或等价无穷大}
    \qitem{ex:ch01-06-02} \difficulty{3} 计算 $\lim_{x \to \infty} \frac{x}{\sqrt{1 + x} - 1}$ \exercisesource{考研数学一·2021}
    \matchexampoint{等价无穷小代换}
    \exercisetip{分子分母同乘共轭式}
\end{enumerate}

\section{函数的连续性}

\subsection{题型7:函数在某点的连续性判断}

\begin{enumerate}[label={【\arabic*】}]
    \qitem{ex:ch01-07-01} \difficulty{1} 判断 $f(x) = \frac{x^2 - 1}{x - 1}$ 在 $x = 1$ 处的连续性 \exercisesource{考研数学一·2018}
    \matchexampoint{可去间断点}
    \exercisetip{利用极限定义判断}
    \qitem{ex:ch01-07-02} \difficulty{1} 判断 $f(x) = \sin x$ 在 $\mathbb{R}$ 上的连续性 \exercisesource{考研数学一·2019}
    \matchexampoint{连续函数}
    \exercisetip{利用三角函数性质}
    \qitem{ex:ch01-07-03} \difficulty{1} 判断 $f(x) = \frac{1}{x}$ 在 $x = 0$ 处的连续性 \exercisesource{考研数学一·2020}
    \matchexampoint{无穷间断点}
    \exercisetip{利用极限定义判断}
\end{enumerate}

\subsection{题型8:间断点类型的判断与分类}

\begin{enumerate}[label={【\arabic*】}]
    \qitem{ex:ch01-08-01} \difficulty{2} 讨论 $f(x) = \frac{x^2 - 4}{x - 2}$ 的连续区间 \exercisesource{考研数学一·2019}
    \matchexampoint{可去间断点}
    \exercisetip{利用极限定义判断}
    \qitem{ex:ch01-08-02} \difficulty{3} 讨论 $f(x) = \frac{1}{x^2 - 1}$ 的间断点类型 \exercisesource{考研数学一·2020}
    \matchexampoint{无穷间断点}
    \exercisetip{利用极限定义判断}
    \qitem{ex:ch01-08-03} \difficulty{2} 判断 $f(x) = \frac{|x|}{x}$ 的连续性 \exercisesource{考研数学一·2021}
    \matchexampoint{跳跃间断点}
    \exercisetip{利用左右极限判断}
\end{enumerate}

\subsection{题型9:分段函数的连续性}

\begin{enumerate}[label={【\arabic*】}]
    \qitem{ex:ch01-09-01} \difficulty{3} 设分段函数 $f(x) = \begin{cases} \frac{x^2 - 1}{x - 1}, & x \neq 1 \\ k, & x = 1 \end{cases}$,求 $k$ 使得 $f(x)$ 在 $x = 1$ 处连续 \exercisesource{考研数学一·2016}
    \matchexampoint{可去间断点}
    \exercisetip{利用连续函数定义,左右极限相等且等于函数值}
    \qitem{ex:ch01-09-02} \difficulty{4} 讨论分段函数 $f(x) = \begin{cases} \sin x, & x < 0 \\ x, & x \geq 0 \end{cases}$ 在 $x = 0$ 处的连续性与可导性 \exercisesource{考研数学一·2017}
    \matchexampoint{连续函数}
    \exercisetip{分别计算左右极限和函数值}
\end{enumerate}

\subsection{题型10:函数图像与极限分析}

\begin{enumerate}[label={【\arabic*】}]
    \qitem{ex:ch01-10-01} \difficulty{3} 观察下图中函数 $y = \sin(x)$ 与 $y = x - \frac{x^3}{6}$ 的图像,分析当 $x \to 0$ 时两者的极限关系,并计算 $\lim_{x \to 0} \frac{\sin x - (x - \frac{x^3}{6})}{x^5}$ \exercisesource{考研数学一·2023}
    \begin{figure}[h]
        \centering
        \includegraphics[width=0.8\textwidth]{images/function_comparison.pdf}
        \caption{函数图像比较}
        \label{fig:function_comparison}
    \end{figure}
    \matchexampoint{泰勒公式}
    \exercisetip{利用泰勒展开分析两者的差异}
    
    \qitem{ex:ch01-10-02} \difficulty{4} 如图所示,三角形 $ABC$ 的三个顶点坐标分别为 $A(0,0)$、$B(4,0)$、$C(1,3)$,求其内切圆的面积,并证明内切圆与三边都相切 \exercisesource{考研数学一·2022}
    \begin{figure}[h]
        \centering
        \includegraphics[width=0.6\textwidth]{images/triangle_circle.pdf}
        \caption{三角形及其内切圆}
        \label{fig:triangle_circle}
    \end{figure}
    \matchexampoint{平面几何}
    \exercisetip{利用平面几何知识计算内切圆半径}
    
    \qitem{ex:ch01-10-03} \difficulty{5} 已知函数 $f(x) = e^x$ 和 $g(x) = 1 + x + \frac{x^2}{2}$,观察下图中两者的图像,分析当 $x \to 0$ 时的极限关系,并计算 $\lim_{x \to 0} \frac{e^x - (1 + x + \frac{x^2}{2})}{x^3}$ \exercisesource{考研数学一·2024}
    \begin{figure}[h]
        \centering
        \includegraphics[width=0.8\textwidth]{images/function_comparison.pdf}
        \caption{函数图像比较(扩展)}
        \label{fig:function_comparison_ext}
    \end{figure}
    \matchexampoint{泰勒公式}
    \exercisetip{利用泰勒展开分析高阶无穷小}
\end{enumerate}

% ========== 本章答案与解析 ==========
\begin{chapter-answers}

    \begin{type-answers}{题型1:极限的四则运算法则}
        \begin{answer-item}{ex:ch01-01-01}
            $\lim_{x \to 0} \frac{x^2 - 1}{x - 1} = \lim_{x \to 0} (x + 1) = 1$(直接代入)
        \end{answer-item}
        \begin{answer-item}{ex:ch01-01-02}
            $\lim_{x \to 2} \frac{x^2 - 4}{x - 2} = \lim_{x \to 2} (x + 2) = 4$(因式分解后化简)
        \end{answer-item}
        \begin{answer-item}{ex:ch01-01-03}
            $\lim_{x \to 2} \frac{x^2 + 3x - 10}{x - 2} = \lim_{x \to 2} (x + 5) = 7$(因式分解后化简)
        \end{answer-item}
    \end{type-answers}
    
    \begin{type-answers}{题型2:等价无穷小替换}
        \begin{answer-item}{ex:ch01-02-01}
            $\lim_{x \to 0} \frac{\sin 2x}{\sin x} = \lim_{x \to 0} \frac{2\sin x \cos x}{\sin x} = \lim_{x \to 0} 2\cos x = 2$(倍角公式化简)
        \end{answer-item}
        \begin{answer-item}{ex:ch01-02-02}
            $\lim_{x \to 0} \frac{e^x - 1}{x} = 1$(等价无穷小代换:$e^x - 1 \sim x$)
        \end{answer-item}
        \begin{answer-item}{ex:ch01-02-03}
            $\lim_{x \to 0} \frac{\ln(1 + x)}{x} = 1$(等价无穷小代换:$\ln(1 + x) \sim x$)
        \end{answer-item}
    \end{type-answers}
    
    \begin{type-answers}{题型3:洛必达法则}
        \begin{answer-item}{ex:ch01-03-01}
            $\lim_{x \to 1} \frac{e^x - e}{x - 1} = \lim_{x \to 1} e^x = e$(直接应用洛必达法则)
        \end{answer-item}
        \begin{answer-item}{ex:ch01-03-02}
            $\lim_{x \to 1} \frac{\ln x}{x - 1} = \lim_{x \to 1} \frac{1/x}{1} = 1$(应用洛必达法则)
        \end{answer-item}
        \begin{answer-item}{ex:ch01-03-03}
            $\lim_{x \to \infty} \frac{x^2}{\ln x} = \lim_{x \to \infty} \frac{2x}{1/x} = \lim_{x \to \infty} 2x^2 = +\infty$(应用洛必达法则,无穷大与无穷大的比较)
        \end{answer-item}
    \end{type-answers}
    
    \begin{type-answers}{题型4:泰勒公式}
        \begin{answer-item}{ex:ch01-04-01}
            利用泰勒展开:$\sin x = x - \frac{x^3}{6} + o(x^3)$,因此
            $\lim_{x \to 0} \frac{x - \sin x}{x^3} = \lim_{x \to 0} \frac{x - (x - \frac{x^3}{6} + o(x^3))}{x^3} = \lim_{x \to 0} \frac{\frac{x^3}{6} + o(x^3)}{x^3} = \frac{1}{6}$
        \end{answer-item}
        \begin{answer-item}{ex:ch01-04-02}
            利用泰勒展开:$\sin x = x - \frac{x^3}{6} + o(x^3)$,因此
            $\lim_{x \to 0} \frac{\sin x - x}{x^3} = \lim_{x \to 0} \frac{(x - \frac{x^3}{6} + o(x^3)) - x}{x^3} = -\frac{1}{6}$
        \end{answer-item}
        \begin{answer-item}{ex:ch01-04-03}
            利用泰勒展开:$\tan x = x + \frac{x^3}{3} + o(x^3)$,$\sin x = x - \frac{x^3}{6} + o(x^3)$,因此
            $\lim_{x \to 0} \frac{\tan x - \sin x}{x^3} = \lim_{x \to 0} \frac{(x + \frac{x^3}{3} + o(x^3)) - (x - \frac{x^3}{6} + o(x^3))}{x^3} = \frac{1}{2}$
        \end{answer-item}
    \end{type-answers}
    
    \begin{type-answers}{题型5:重要极限公式}
        \begin{answer-item}{ex:ch01-05-01}
            $\lim_{x \to 0} \frac{\sin x}{x} = 1$(直接应用重要极限公式)
        \end{answer-item}
        \begin{answer-item}{ex:ch01-05-02}
            $\lim_{x \to \infty} (1 + \frac{1}{x})^x = e$(直接应用重要极限公式)
        \end{answer-item}
        \begin{answer-item}{ex:ch01-05-03}
            $\lim_{x \to \infty} (1 + \frac{2}{x})^x = \lim_{x \to \infty} \left[(1 + \frac{2}{x})^{x/2}\right]^2 = e^2$(应用重要极限公式)
        \end{answer-item}
    \end{type-answers}
    
    \begin{type-answers}{题型6:无穷大的比较}
        \begin{answer-item}{ex:ch01-06-01}
            $\lim_{x \to \infty} \frac{x^2}{\ln x} = +\infty$(应用洛必达法则,无穷大与无穷大的比较,$x^2$ 是比 $\ln x$ 高阶的无穷大)
        \end{answer-item}
        \begin{answer-item}{ex:ch01-06-02}
            $\lim_{x \to \infty} \frac{x}{\sqrt{1 + x} - 1} = \lim_{x \to \infty} \frac{x(\sqrt{1 + x} + 1)}{(\sqrt{1 + x} - 1)(\sqrt{1 + x} + 1)} = \lim_{x \to \infty} \frac{x(\sqrt{1 + x} + 1)}{x} = \lim_{x \to \infty} (\sqrt{1 + x} + 1) = +\infty$(分子分母同乘共轭式)
        \end{answer-item}
    \end{type-answers}
    
    \begin{type-answers}{题型7:函数在某点的连续性判断}
        \begin{answer-item}{ex:ch01-07-01}
            $f(x) = \frac{x^2 - 1}{x - 1}$ 在 $x = 1$ 处无定义,因此不连续。但 $\lim_{x \to 1} f(x) = \lim_{x \to 1} (x + 1) = 2$,所以 $x = 1$ 是可去间断点。
        \end{answer-item}
        \begin{answer-item}{ex:ch01-07-02}
            $\sin x$ 在 $\mathbb{R}$ 上连续,因为对于任意 $x_0 \in \mathbb{R}$,有 $\lim_{x \to x_0} \sin x = \sin x_0$。
        \end{answer-item}
        \begin{answer-item}{ex:ch01-07-03}
            $f(x) = \frac{1}{x}$ 在 $x = 0$ 处无定义,且 $\lim_{x \to 0^+} f(x) = +\infty$,$\lim_{x \to 0^-} f(x) = -\infty$,因此 $x = 0$ 是无穷间断点,$f(x)$ 在 $x = 0$ 处不连续。
        \end{answer-item}
    \end{type-answers}
    
    \begin{type-answers}{题型8:间断点类型的判断与分类}
        \begin{answer-item}{ex:ch01-08-01}
            $f(x) = \frac{x^2 - 4}{x - 2} = x + 2$($x \neq 2$),因此连续区间为 $(-\infty, 2) \cup (2, +\infty)$,$x = 2$ 是可去间断点。
        \end{answer-item}
        \begin{answer-item}{ex:ch01-08-02}
            $f(x) = \frac{1}{x^2 - 1} = \frac{1}{(x - 1)(x + 1)}$,在 $x = 1$ 和 $x = -1$ 处无定义,且 $\lim_{x \to 1} f(x) = \lim_{x \to -1} f(x) = \infty$,因此 $x = 1$ 和 $x = -1$ 都是无穷间断点。
        \end{answer-item}
        \begin{answer-item}{ex:ch01-08-03}
            $f(x) = \frac{|x|}{x} = \begin{cases} 1, & x > 0 \\ -1, & x < 0 \end{cases}$,在 $x = 0$ 处无定义,且 $\lim_{x \to 0^+} f(x) = 1$,$\lim_{x \to 0^-} f(x) = -1$,因此 $x = 0$ 是跳跃间断点,$f(x)$ 的连续区间为 $(-\infty, 0) \cup (0, +\infty)$。
        \end{answer-item}
    \end{type-answers}
    
    \begin{type-answers}{题型9:分段函数的连续性}
        \begin{answer-item}{ex:ch01-09-01}
            $\lim_{x \to 1} f(x) = \lim_{x \to 1} \frac{x^2 - 1}{x - 1} = \lim_{x \to 1} (x + 1) = 2$,要使 $f(x)$ 在 $x = 1$ 处连续,需 $f(1) = k = \lim_{x \to 1} f(x) = 2$,因此 $k = 2$。
        \end{answer-item}
        \begin{answer-item}{ex:ch01-09-02}
            连续性:$\lim_{x \to 0^-} f(x) = \lim_{x \to 0^-} \sin x = 0$,$\lim_{x \to 0^+} f(x) = \lim_{x \to 0^+} x = 0$,且 $f(0) = 0$,因此 $f(x)$ 在 $x = 0$ 处连续。
            
            可导性:左导数 $f'_-(0) = \lim_{h \to 0^-} \frac{f(0 + h) - f(0)}{h} = \lim_{h \to 0^-} \frac{\sin h - 0}{h} = 1$,右导数 $f'_+(0) = \lim_{h \to 0^+} \frac{f(0 + h) - f(0)}{h} = \lim_{h \to 0^+} \frac{h - 0}{h} = 1$,左导数等于右导数,因此 $f(x)$ 在 $x = 0$ 处可导。
        \end{answer-item}
    \end{type-answers}
    
    \begin{type-answers}{题型10:函数图像与极限分析}
        \begin{answer-item}{ex:ch01-10-01}
            当 $x \to 0$ 时,$\sin x$ 和 $x - \frac{x^3}{6}$ 都是无穷小量,且两者的差为高阶无穷小。利用泰勒展开:
            $\sin x = x - \frac{x^3}{6} + \frac{x^5}{120} + o(x^5)$
            因此,$\lim_{x \to 0} \frac{\sin x - (x - \frac{x^3}{6})}{x^5} = \lim_{x \to 0} \frac{\frac{x^5}{120} + o(x^5)}{x^5} = \frac{1}{120}$
        \end{answer-item}
        \begin{answer-item}{ex:ch01-10-02}
            首先计算三角形的边长:
            $a = BC = \sqrt{(4-1)^2 + (0-3)^2} = \sqrt{9 + 9} = \sqrt{18} = 3\sqrt{2}$
            $b = AC = \sqrt{(1-0)^2 + (3-0)^2} = \sqrt{1 + 9} = \sqrt{10}$
            $c = AB = \sqrt{(4-0)^2 + (0-0)^2} = 4$
            
            半周长 $s = \frac{a + b + c}{2} = \frac{3\sqrt{2} + \sqrt{10} + 4}{2}$
            
            面积 $S = \frac{1}{2} \times 底 \times 高 = \frac{1}{2} \times 4 \times 3 = 6$
            
            内切圆半径 $r = \frac{S}{s} = \frac{6}{\frac{3\sqrt{2} + \sqrt{10} + 4}{2}} = \frac{12}{3\sqrt{2} + \sqrt{10} + 4}$
            
            内切圆面积 $A = \pi r^2 = \pi \left( \frac{12}{3\sqrt{2} + \sqrt{10} + 4} \right)^2$
        \end{answer-item}
        \begin{answer-item}{ex:ch01-10-03}
            当 $x \to 0$ 时,$e^x$ 和 $1 + x + \frac{x^2}{2}$ 都是无穷小量,且两者的差为高阶无穷小。利用泰勒展开:
            $e^x = 1 + x + \frac{x^2}{2} + \frac{x^3}{6} + o(x^3)$
            因此,$\lim_{x \to 0} \frac{e^x - (1 + x + \frac{x^2}{2})}{x^3} = \lim_{x \to 0} \frac{\frac{x^3}{6} + o(x^3)}{x^3} = \frac{1}{6}$
        \end{answer-item}
    \end{type-answers}

\end{chapter-answers}

