% 第1章:绪论
\chapter{绪论}
\label{ch:intro}

\section{考研数学一概述}

考研数学一是全国硕士研究生入学统一考试中的重要科目,主要考查考生的数学基础知识、基本技能和基本方法,以及运用这些知识解决实际问题的能力。数学一的考试内容包括高等数学、线性代数、概率论与数理统计三个部分,其中高等数学占56%,线性代数占22%,概率论与数理统计占22%。

\section{考研数学一的考试特点}

\begin{itemize}
    \item \textbf{知识点覆盖广}:考试内容涵盖高等数学、线性代数、概率论与数理统计的主要知识点
    \item \textbf{注重基础}:强调对基本概念、基本理论和基本方法的掌握
    \item \textbf{强调应用}:注重考查运用数学知识解决实际问题的能力
    \item \textbf{题型固定}:主要包括选择题、填空题和解答题三种题型
    \item \textbf{难度适中}:既有基础题,也有一定难度的综合题
\end{itemize}

\section{考研数学一的复习方法}

\begin{itemize}
    \item \textbf{系统复习}:按照考试大纲,系统复习所有知识点
    \item \textbf{多做习题}:通过大量练习,巩固所学知识,提高解题能力
    \item \textbf{总结归纳}:总结解题方法和技巧,形成自己的知识体系
    \item \textbf{模拟考试}:定期进行模拟考试,适应考试节奏,提高应试能力
    \item \textbf{错题整理}:建立错题集,分析错误原因,避免重复犯错
\end{itemize}

\section{本书的编写目的和使用方法}

本书是一本针对考研数学一的习题集与错题集,旨在帮助考生系统复习数学一的知识点,提高解题能力,应对考试。本书的主要特点包括:

\begin{itemize}
    \item \textbf{章节结构清晰}:按照考研数学一的知识点,分章节组织习题
    \item \textbf{题型全面}:涵盖考研数学一的各种题型
    \item \textbf{难度分级}:习题难度分为1-5级,适合不同阶段的复习
    \item \textbf{答案详细}:每个习题都有详细的解析,包括易错点和一题多解
    \item \textbf{双向超链接}:习题和答案之间可以相互跳转,方便使用
\end{itemize}

\section{如何使用本书}

\begin{itemize}
    \item \textbf{基础阶段}:重点做难度1-3级的习题,巩固基础知识
    \item \textbf{强化阶段}:重点做难度3-5级的习题,提高解题能力
    \item \textbf{冲刺阶段}:做模拟题和历年真题,适应考试节奏
    \item \textbf{错题整理}:将做错的习题整理到错题集,定期复习
    \item \textbf{反复练习}:对于重要的知识点和题型,反复练习,直到熟练掌握
\end{itemize}