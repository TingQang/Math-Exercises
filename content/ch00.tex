\chapter*{前言}

\section{使用说明}

本笔记是基于2025考研数学一大纲核心知识与对应题型总结编写的习题集与错题集,旨在帮助考生系统复习考研数学一的核心知识点,掌握各类题型的解题方法和技巧。

\section{编写目的}

- 提供系统化的考研数学一知识点总结
- 涵盖各类题型的解题方法和技巧
- 帮助考生建立清晰的知识体系
- 便于考生进行针对性的练习和复习
- 提供错题记录和复盘功能

\section{核心特色}

1. **结构化组织**:采用"章-节-题型"的结构化组织方式,便于知识点的系统复习和查找。

2. **难度分级**:每道习题都标注了难度等级(1-5级),便于考生根据自身情况选择合适的习题进行练习。

3. **题型归类**:将习题按照题型进行归类,便于考生掌握各类题型的解题方法和技巧。

4. **知识点关联**:每道习题都标注了考查的核心知识点,便于考生进行针对性的复习和强化。

5. **来源标注**:部分习题标注了来源(如真题年份、辅导书名称),便于考生了解命题趋势和参考价值。

\section{使用方法}

1. **系统复习**:按照章节顺序系统复习知识点,掌握各类题型的解题方法和技巧。

2. **针对性练习**:根据自身情况选择合适难度的习题进行练习,重点强化薄弱环节。

3. **错题记录**:将做错的习题记录在错题集中,分析错误原因,总结解题经验。

4. **定期复盘**:定期回顾错题集,巩固知识点,提高解题能力。

5. **模拟测试**:在复习后期,进行模拟测试,熟悉考试题型和节奏,提高应试能力。

\section{格式说明}

- 章节标题:使用\verb|\chapter|命令,居中、加粗显示。
- 小节标题:使用\verb|\section|命令,左对齐、加粗显示。
- 题型标题:使用\verb|\pt|命令,居中、加粗显示,格式为"题型X:题型名称"。
- 习题:使用\verb|exercise|环境,包含难度等级、题目内容、来源标注、考查知识点等。
- 答案:使用\verb|solution|环境,包含详细解题过程和易错点分析。

\section{符号说明}

- \textbf{【\textcolor{kaoyan-red}{考点}】}:标注习题考查的核心考点。
- \textbf{【\textcolor{kaoyan-blue}{重点}】}:标注习题对应的重点知识点。
- \textbf{【\textcolor{gray}{来源}】}:标注题目来源。
- \textbf{【\textcolor{kaoyan-blue}{匹配考点}】}:标注题目与章节考点的匹配关系。
- \textbf{【\textcolor{kaoyan-orange}{提示}】}:提供解题思路提示。
- \textbf{【\textcolor{kaoyan-orange}{解法X}】}:标注一题多解的解法名称。

