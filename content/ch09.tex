\chapter{矩阵}

\section{考点说明}
\begin{itemize}
\item 矩阵的概念;
\item 矩阵的线性运算(加法、数乘);
\item 矩阵的乘法;
\item 逆矩阵的性质;
\item 矩阵的初等变换;
\item 等价矩阵;
\item 矩阵的秩与初等变换的关系
\item 矩阵的转置;
\item 逆矩阵的概念;
\item 逆矩阵的求法(伴随矩阵法、初等行变换法);
\item 分块矩阵的逆矩阵
\item 矩阵的秩;
\item 初等矩阵的概念与性质;
\item 方阵的行列式
\item 方阵的幂;
\end{itemize}

\section{第一节 矩阵的概念与运算}

\pt{矩阵的线性运算与乘法运算;}

\pt{矩阵转置的性质应用;}

\pt{方阵幂的计算;}

\pt{方阵行列式的计算;}

\pt{矩阵乘法的特殊性应用(不交换性);}

\pt{分块矩阵的运算;}

\pt{方阵幂的递推公式求解;}

\pt{伴随矩阵的运算性质应用;}

\pt{含参数矩阵的运算与讨论}

\section{第二节 逆矩阵}

\pt{逆矩阵的存在性判断;}

\pt{逆矩阵的计算(伴随矩阵法、初等行变换法);}

\pt{逆矩阵性质的应用;}

\pt{分块矩阵逆矩阵的求解;}

\pt{利用逆矩阵求解矩阵方程;}

\pt{逆矩阵与伴随矩阵的关系应用;}

\pt{含参数矩阵的可逆性讨论;}

\pt{逆矩阵的证明题(如证明AB可逆则A、B可逆)}

\section{第三节 初等变换与初等矩阵}

\pt{矩阵的初等变换;}

\pt{初等矩阵与初等变换的关系;}

\pt{矩阵秩的计算与判断;}

\pt{等价矩阵的判断与应用;}

\pt{利用初等变换求逆矩阵或解矩阵方程;}

\pt{初等矩阵的乘积运算;}

\pt{矩阵秩的不等式应用;}

\pt{含参数矩阵的秩的讨论;}

\pt{等价标准形的应用;}

\pt{矩阵秩与初等变换关系的综合应用}

