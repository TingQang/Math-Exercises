\chapter{矩阵的特征值和特征向量}

\section{考点说明}
\begin{itemize}
\item 特征值与特征向量的性质;
\item 特征方程与特征多项式
\item 矩阵可对角化的充要条件;
\item 相似矩阵的概念与性质;
\item 特征值与特征向量的概念;
\item 实对称矩阵的特征值与特征向量的性质;
\item 实对称矩阵的对角化
\end{itemize}

\section{第一节 特征值与特征向量的概念与性质}

\pt{矩阵特征值与特征向量的求解;}

\pt{特征值与特征向量性质的应用;}

\pt{特征方程与特征多项式的求解;}

\pt{含参数矩阵的特征值与特征向量求解;}

\pt{抽象矩阵的特征值与特征向量求解;}

\pt{特征值与矩阵秩的关系应用;}

\pt{特征向量的线性无关性证明;}

\pt{利用特征值判断矩阵可逆性}

\section{第二节 相似矩阵与矩阵的对角化}

\pt{相似矩阵的判断与性质应用;}

\pt{矩阵可对角化的判断与对角化过程;}

\pt{实对称矩阵的对角化;}

\pt{利用相似矩阵求方阵的幂;}

\pt{含参数矩阵的可对角化判断;}

\pt{相似矩阵与特征值的关系应用;}

\pt{实对称矩阵特征向量的正交性应用;}

\pt{矩阵对角化的应用(如求解线性微分方程组);}

\pt{相似变换矩阵的求解}

