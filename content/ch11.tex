\chapter{线性方程组}

\section{考点说明}
\begin{itemize}
\item 线性方程组解的几何意义(数一重点)
\item 解的结构(基础解系、通解)
\item 齐次线性方程组有非零解的充要条件;
\item 线性方程组的表示形式(矩阵形式、向量形式);
\item 非齐次线性方程组有解的充要条件;
\item 线性方程组解的概念
\item 含参数线性方程组的求解;
\end{itemize}

\section{第一节 线性方程组的基本概念}

\pt{线性方程组的矩阵形式与向量形式转化;}

\pt{线性方程组解的验证;}

\pt{线性方程组与向量组线性表示的关系应用;}

\pt{含参数线性方程组的表示形式转化;}

\pt{齐次与非齐次线性方程组的表示形式关联}

\section{第二节 线性方程组的解的判定}

\pt{齐次线性方程组有非零解的判断与基础解系求解;}

\pt{非齐次线性方程组有解的判断与通解求解;}

\pt{线性方程组解的存在性与向量组线性相关性的关系;}

\pt{含参数线性方程组解的存在性讨论;}

\pt{基础解系的性质应用(如证明基础解系线性无关);}

\pt{非齐次线性方程组解的结构应用(通解的构造);}

\pt{两个线性方程组同解或公共解的求解;}

\pt{线性方程组解的空间维数计算}

\section{第三节 线性方程组的应用}

\pt{含参数线性方程组的求解;}

\pt{线性方程组解的几何意义应用;}

\pt{含参数线性方程组解的情况分类讨论;}

\pt{利用线性方程组求解向量线性表示问题;}

\pt{线性方程组在几何中的应用(如求平面交点、直线位置关系);}

\pt{线性方程组解的应用证明题}

