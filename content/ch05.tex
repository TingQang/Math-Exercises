\chapter{多元函数微分学}

\section{考点说明}
\begin{itemize}
\item 空间曲线的切线和法平面;
\item 多元函数的极值;
\item 多元复合函数、隐函数的求导法则;
\item 多元函数的概念;
\item 条件极值与拉格朗日乘数法
\item 曲面的切平面和法线;
\item 多元函数的偏导数;
\item 二阶偏导数;
\item 有界闭区域上多元连续函数的性质
\item 二元函数的极限与连续的概念;
\item 全微分;
\item 多元函数的最大值与最小值;
\item 二元函数的几何意义;
\item 二元函数的二阶泰勒公式
\item 方向导数与梯度;
\item 全微分存在的必要条件和充分条件;
\end{itemize}

\section{第一节 多元函数的概念与极限连续}

\pt{二元函数的定义域求解;}

\pt{二元函数极限的存在性判断与计算;}

\pt{多元函数连续性的判断;}

\pt{有界闭区域上多元连续函数性质的应用;}

\pt{二元函数极限不存在的证明;}

\pt{含参数多元函数的连续性讨论;}

\pt{利用多元连续函数性质证明方程根的存在性}

\section{第二节 偏导数与全微分}

\pt{偏导数(一阶、二阶)的计算;}

\pt{全微分的计算与全微分存在性判断;}

\pt{多元复合函数求导(含抽象函数);}

\pt{隐函数求导(方程和方程组确定的隐函数);}

\pt{方向导数与梯度的计算;}

\pt{空间曲线的切线和法平面方程求解;}

\pt{曲面的切平面和法线方程求解;}

\pt{抽象复合函数的高阶偏导数计算;}

\pt{隐函数组的偏导数计算;}

\pt{梯度的几何意义应用(求函数增长最快方向);}

\pt{全微分在近似计算中的应用;}

\pt{含参数的多元函数偏导数讨论;}

\pt{二元函数二阶泰勒公式的应用}

\section{第三节 多元函数的极值与最值}

\pt{多元函数极值的判定与求解;}

\pt{多元函数在有界闭区域上的最大值与最小值求解;}

\pt{条件极值的求解(拉格朗日乘数法);}

\pt{多元函数极值的应用;}

\pt{含参数多元函数的极值讨论;}

\pt{利用极值证明不等式;}

\pt{多约束条件下的条件极值求解;}

\pt{极值点的充分条件与必要条件综合应用}

