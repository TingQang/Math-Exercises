\chapter{无穷级数}

\section{考点说明}
\begin{itemize}
\item 绝对收敛与条件收敛的概念及判断
\item 函数的幂级数展开式;
\item 幂级数的收敛半径、收敛区间和收敛域;
\item 傅里叶级数的概念;
\item 狄利克雷(Dirichlet)收敛定理;
\item 交错级数的审敛法(莱布尼茨判别法);
\item 函数在[0, l]上的正弦级数和余弦级数
\item 常数项级数的概念;
\item 级数的基本性质;
\item 幂级数的和函数;
\item 级数的收敛与发散的定义;
\item 幂级数的四则运算与复合运算
\item 幂级数的概念;
\item 正项级数的审敛法(比较审敛法、比值审敛法、根值审敛法);
\item 幂级数在收敛区间内的基本性质;
\item 级数收敛的必要条件
\item 函数在[-l, l]上的傅里叶级数;
\end{itemize}

\section{第一节 常数项级数的概念与性质}

\pt{常数项级数收敛性的判断;}

\pt{利用级数性质判断收敛性;}

\pt{级数收敛必要条件的应用;}

\pt{利用级数收敛性判断数列极限;}

\pt{含参数级数的收敛性讨论;}

\pt{级数敛散性的反例构造}

\section{第二节 常数项级数的审敛法}

\pt{正项级数的审敛;}

\pt{交错级数的审敛(莱布尼茨判别法);}

\pt{绝对收敛与条件收敛的判断;}

\pt{正项级数审敛法的综合应用(比较与比值结合);}

\pt{交错级数余项的估计;}

\pt{任意项级数的敛散性判断;}

\pt{含参数级数的绝对收敛与条件收敛判断}

\section{第三节 幂级数}

\pt{幂级数收敛半径、收敛区间、收敛域的求解;}

\pt{幂级数和函数的求解;}

\pt{函数的幂级数展开;}

\pt{幂级数在收敛域内的逐项求导与逐项积分应用;}

\pt{含参数幂级数的收敛域求解;}

\pt{利用幂级数和函数求解常数项级数的和;}

\pt{函数的幂级数展开在近似计算中的应用;}

\pt{幂级数的四则运算应用;}

\pt{幂级数复合运算的收敛域求解}

\section{第四节 傅里叶级数}

\pt{傅里叶系数的计算;}

\pt{傅里叶级数的收敛性判断(狄利克雷定理应用);}

\pt{函数的傅里叶级数展开(正弦级数、余弦级数);}

\pt{利用傅里叶级数求常数项级数的和;}

\pt{非周期函数的周期延拓与傅里叶展开;}

\pt{傅里叶级数在特定点的收敛值计算}

