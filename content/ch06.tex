% 第6章:级数
\chapter{级数}
\label{ch:series}

本章介绍无穷级数的概念、性质、收敛性判别法,以及幂级数和傅里叶级数的理论与应用。

% 题型6.1 数项级数的收敛性
\begin{question-type}{数项级数的收敛性}
本题型主要考查数项级数的收敛性判别,包括正项级数判别法、交错级数判别法、一般项级数判别法。

\begin{exercise}
判断正项级数的收敛性:
\begin{enumerate}[label={【\arabic*】}]
    \item \difficulty{1} $\sum_{n=1}^\infty \frac{1}{n^2}$ \exercisesource{级数基础}
    \item \difficulty{2} $\sum_{n=1}^\infty \frac{n}{n^2 + 1}$ \exercisesource{李永乐复习全书P501}
    \item \difficulty{3} $\sum_{n=1}^\infty \frac{1}{n \ln n}$ \exercisesource{2024考研数学一真题}
    \item \difficulty{2} $\sum_{n=1}^\infty \frac{1}{n^p}$(p级数) \exercisesource{B站 BV1xx4y1E7xx}
\end{enumerate}
\end{exercise}

\begin{exercise}
判断交错级数的收敛性:
\begin{enumerate}[label={【\arabic*】}]
    \item \difficulty{2} $\sum_{n=1}^\infty (-1)^{n-1} \frac{1}{n}$ \exercisesource{交错级数}
    \item \difficulty{3} $\sum_{n=1}^\infty (-1)^{n-1} \frac{n}{n^2 + 1}$ \exercisesource{李永乐复习全书P515}
    \item \difficulty{4} $\sum_{n=1}^\infty (-1)^{n-1} \frac{\ln n}{n}$ \exercisesource{2023考研数学一真题}
\end{enumerate}
\end{exercise}

\begin{exercise}
使用一般项级数判别法:
\begin{enumerate}[label={【\arabic*】}]
    \item \difficulty{3} $\sum_{n=1}^\infty \frac{(-1)^n}{\sqrt{n}}$ \exercisesource{一般项判别}
    \item \difficulty{4} $\sum_{n=1}^\infty \frac{\sin n}{n^2}$ \exercisesource{B站 BV1xx4y1E7xx}
\end{enumerate}
\end{exercise}
\end{question-type}

% 题型6.2 幂级数
\begin{question-type}{幂级数}
本题型主要考查幂级数的收敛域、性质、展开式以及应用。

\begin{exercise}
求幂级数的收敛域:
\begin{enumerate}[label={【\arabic*】}]
    \item \difficulty{1} $\sum_{n=0}^\infty x^n$ \exercisesource{幂级数基础}
    \item \difficulty{2} $\sum_{n=0}^\infty \frac{x^n}{n!}$ \exercisesource{李永乐复习全书P535}
    \item \difficulty{3} $\sum_{n=1}^\infty \frac{x^n}{n \cdot 2^n}$ \exercisesource{2024考研数学一真题}
\end{enumerate}
\end{exercise}

\begin{exercise}
将函数展开为幂级数:
\begin{enumerate}[label={【\arabic*】}]
    \item \difficulty{2} $e^x$ 在 $x = 0$ 处的泰勒展开 \exercisesource{泰勒展开}
    \item \difficulty{3} $\ln(1 + x)$ 在 $x = 0$ 处的展开 \exercisesource{李永乐复习全书P548}
    \item \difficulty{2} $\sin x$ 在 $x = 0$ 处的展开 \exercisesource{B站 BV1xx4y1E7xx}
\end{enumerate}
\end{exercise}

\begin{exercise}
利用幂级数求极限或定积分:
\begin{enumerate}[label={【\arabic*】}]
    \item \difficulty{3} $\lim_{x \to 0} \frac{e^x - 1 - x}{x^2}$ \exercisesource{幂级数求极限}
    \item \difficulty{4} $\int_0^1 \frac{\sin x}{x} \dx$ \exercisesource{2023考研数学一真题}
\end{enumerate}
\end{exercise}
\end{question-type}

% 题型6.3 傅里叶级数
\begin{question-type}{傅里叶级数}
本题型主要考查周期函数的傅里叶级数展开及其收敛性。

\begin{exercise}
求函数的傅里叶级数:
\begin{enumerate}[label={【\arabic*】}]
    \item \difficulty{3} $f(x) = x$ 在 $(-\pi, \pi)$ 上的傅里叶级数 \exercisesource{傅里叶基础}
    \item \difficulty{4} $f(x) = |x|$ 在 $(-\pi, \pi)$ 上的傅里叶级数 \exercisesource{李永乐复习全书P565}
    \item \difficulty{4} $f(x) = \begin{cases} 1 & 0 < x < \pi \\ 0 & -\pi < x < 0 \end{cases}$ 的傅里叶级数 \exercisesource{2024考研数学一真题}
\end{enumerate}
\end{exercise}

\begin{exercise}
讨论傅里叶级数的收敛性:
\begin{enumerate}[label={【\arabic*】}]
    \item \difficulty{3} 周期为 $2\pi$ 的函数在间断点的收敛性 \exercisesource{收敛性讨论}
    \item \difficulty{4} 傅里叶级数在连续点和间断点的值 \exercisesource{B站 BV1xx4y1E7xx}
\end{enumerate}
\end{exercise}
\end{question-type} 
