\chapter{数理统计的基本概念}

\section{考点说明}
\begin{itemize}
\item 统计量的概念;
\item 简单随机样本的性质;
\item 样本观测值与样本函数
\item 总体与个体的概念;
\item 三大抽样分布(χ²分布、t分布、F分布)的概念与性质;
\item 正态总体下的抽样分布定理;
\item 经验分布函数
\item 样本的概念与样本空间;
\item 常用统计量(样本均值、样本方差、样本标准差、样本矩);
\end{itemize}

\section{第一节 总体、个体与样本}

\pt{简单随机样本的判定与性质应用;}

\pt{样本函数的构造与计算;}

\pt{样本空间的确定;}

\pt{样本均值、样本方差的初步计算;}

\pt{简单随机样本的独立性应用;}

\pt{样本观测值的整理与应用}

\section{第二节 统计量及其分布}

\pt{统计量的判定;}

\pt{样本均值、样本方差等常用统计量的计算;}

\pt{三大抽样分布的判定与性质应用;}

\pt{正态总体下抽样分布定理的应用;}

\pt{统计量的分布求解;}

\pt{三大抽样分布的分位数计算与应用;}

\pt{经验分布函数的构造与性质应用}

