% 第4章:多元函数微分学
\chapter{多元函数微分学}
\label{ch:multivariate-differential}

本章介绍多元函数的极限、连续、偏导数、全微分、多元复合函数求导以及多元函数的极值问题。

% 题型4.1 偏导数与全微分
\begin{question-type}{偏导数与全微分}
本题型主要考查多元函数偏导数的计算、全微分的概念与计算,以及可微的条件。

\begin{exercise}
计算二元函数的偏导数:
\begin{enumerate}[label={【\arabic*】}]
    \item \difficulty{1} $z = x^2 + y^2$ 的偏导数 \exercisesource{偏导数基础}
    \item \difficulty{2} $z = e^{xy}$ 的 $\frac{\partial z}{\partial x}$ 和 $\frac{\partial z}{\partial y}$ \exercisesource{李永乐复习全书P301}
    \item \difficulty{3} $z = \sin(x^2 + y^2)$ 的二阶偏导数 \exercisesource{2024考研数学一真题}
\end{enumerate}
\end{exercise}

\begin{exercise}
计算全微分:
\begin{enumerate}[label={【\arabic*】}]
    \item \difficulty{2} $z = x^3 + y^3 - 3xy$ 的全微分 \exercisesource{全微分计算}
    \item \difficulty{2} $z = \ln(x + y)$ 的全微分 \exercisesource{B站 BV1xx4y1E7xx}
    \item \difficulty{3} $z = \arctan \frac{y}{x}$ 的全微分 \exercisesource{隐函数全微分}
\end{enumerate}
\end{exercise}

\begin{exercise}
判断函数的可微性:
\begin{enumerate}[label={【\arabic*】}]
    \item \difficulty{4} $f(x,y) = \begin{cases} \frac{xy}{x^2 + y^2} & (x,y) \neq (0,0) \\ 0 & (x,y) = (0,0) \end{cases}$ 在 $(0,0)$ 处 \exercisesource{可微性判断}
    \item \difficulty{3} $f(x,y) = |x| + |y|$ 在 $(0,0)$ 处 \exercisesource{李永乐复习全书P315}
\end{enumerate}
\end{exercise}
\end{question-type}

% 题型4.2 多元复合函数求导
\begin{question-type}{多元复合函数求导}
本题型主要考查多元复合函数的偏导数计算,包括链式法则的应用。

\begin{exercise}
计算复合函数的偏导数:
\begin{enumerate}[label={【\arabic*】}]
    \item \difficulty{3} $z = f(u,v), u = x + y, v = xy$,求 $\frac{\partial z}{\partial x}$ \exercisesource{复合函数求导}
    \item \difficulty{2} $z = f(x^2 + y^2)$,求 $\frac{\partial z}{\partial x}$ 和 $\frac{\partial z}{\partial y}$ \exercisesource{李永乐复习全书P325}
    \item \difficulty{3} $z = f(\sin x, \cos y)$,求 $\frac{\partial z}{\partial x}$ 和 $\frac{\partial z}{\partial y}$ \exercisesource{2024考研数学一真题}
\end{enumerate}
\end{exercise}

\begin{exercise}
隐函数的偏导数:
\begin{enumerate}[label={【\arabic*】}]
    \item \difficulty{3} $F(x,y,z) = x^2 + y^2 + z^2 - 1 = 0$,求 $\frac{\partial z}{\partial x}$ 和 $\frac{\partial z}{\partial y}$ \exercisesource{隐函数偏导}
    \item \difficulty{4} $z = f(x,y), F(x,y,z) = 0$,求 $\frac{\partial z}{\partial x}$ \exercisesource{B站 BV1xx4y1E7xx}
\end{enumerate}
\end{exercise}
\end{question-type}

% 题型4.3 多元函数极值与最值
\begin{question-type}{多元函数极值与最值}
本题型主要考查多元函数的无条件极值、有条件极值(拉格朗日乘数法)以及最值问题。

\begin{exercise}
求二元函数的无条件极值:
\begin{enumerate}[label={【\arabic*】}]
    \item \difficulty{2} $z = x^2 + y^2$ 的极值 \exercisesource{极值基础}
    \item \difficulty{3} $z = x^3 + y^3 - 3xy$ 的极值 \exercisesource{李永乐复习全书P345}
    \item \difficulty{3} $z = e^{-(x^2 + y^2)}$ 的极值 \exercisesource{2023考研数学一真题}
\end{enumerate}
\end{exercise}

\begin{exercise}
使用拉格朗日乘数法求条件极值:
\begin{enumerate}[label={【\arabic*】}]
    \item \difficulty{3} 在 $x^2 + y^2 = 1$ 上求 $z = x + y$ 的极值 \exercisesource{拉格朗日乘数法}
    \item \difficulty{4} 在 $x + y + z = 1, x > 0, y > 0, z > 0$ 上求 $xyz$ 的最大值 \exercisesource{B站 BV1xx4y1E7xx}
\end{enumerate}
\end{exercise}

\begin{exercise}
求多元函数在区域上的最值:
\begin{enumerate}[label={【\arabic*】}]
    \item \difficulty{3} $z = x^2 + y^2$ 在 $x^2 + y^2 \leq 1$ 上的最值 \exercisesource{闭区域最值}
    \item \difficulty{2} $z = xy$ 在 $x + y = 1, x > 0, y > 0$ 上的最值 \exercisesource{李永乐复习全书P358}
\end{enumerate}
\end{exercise}
\end{question-type} 
