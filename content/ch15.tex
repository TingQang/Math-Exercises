\chapter{随机变量及其分布}

\section{考点说明}
\begin{itemize}
\item 分布律与分布函数的关系
\item 常见连续型分布(均匀分布、指数分布、正态分布);
\item 随机变量函数的分布
\item 离散型随机变量的分布律;
\item 分布函数的概念与性质;
\item 分布函数的计算
\item 概率密度与分布函数的关系;
\item 随机变量的概念;
\item 连续型随机变量的概率密度;
\item 常见离散型分布(0-1分布、二项分布、泊松分布、几何分布、超几何分布);
\end{itemize}

\section{第一节 随机变量的概念与分布函数}

\pt{分布函数的性质应用;}

\pt{利用分布函数求概率;}

\pt{分布函数的判定(验证是否满足性质);}

\pt{含参数分布函数的求解与讨论;}

\pt{分布函数与随机变量类型的关系判断}

\section{第二节 离散型随机变量}

\pt{离散型随机变量分布律的求解与验证;}

\pt{常见离散型分布的概率计算;}

\pt{由分布律求分布函数;}

\pt{含参数离散型随机变量的分布律求解;}

\pt{离散型随机变量函数的分布求解;}

\pt{常见离散型分布的应用问题(如二项分布的次数计算);}

\pt{离散型随机变量的期望与方差结合计算}

\section{第三节 连续型随机变量}

\pt{连续型随机变量概率密度的性质应用;}

\pt{常见连续型分布的概率计算;}

\pt{由概率密度求分布函数;}

\pt{随机变量函数的分布;}

\pt{含参数连续型随机变量的概率密度求解;}

\pt{连续型随机变量的期望与方差结合计算;}

\pt{常见连续型分布的性质应用(如正态分布的对称性);}

\pt{连续型随机变量的分位数计算;}

\pt{随机变量函数分布的综合问题(如分段函数)}

