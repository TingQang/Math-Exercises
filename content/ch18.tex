\chapter{大数定律和中心极限定理}

\section{考点说明}
\begin{itemize}
\item 切比雪夫不等式;
\item 棣莫弗-拉普拉斯(De Moivre-Laplace)中心极限定理(二项分布以正态分布为极限分布)
\item 列维-林德伯格(Levy-Lindberg)中心极限定理(独立同分布中心极限定理);
\item 辛钦大数定律
\item 伯努利大数定律;
\item 切比雪夫大数定律;
\end{itemize}

\section{第一节 大数定律}

\pt{切比雪夫不等式的应用;}

\pt{大数定律的理解与应用;}

\pt{利用切比雪夫不等式估计概率;}

\pt{大数定律在实际问题中的应用(如频率估计概率);}

\pt{切比雪夫大数定律的条件验证;}

\pt{辛钦大数定律的应用(如样本均值估计期望)}

\section{第二节 中心极限定理}

\pt{利用列维-林德伯格中心极限定理近似计算概率;}

\pt{利用棣莫弗-拉普拉斯中心极限定理近似计算二项分布的概率;}

\pt{中心极限定理的适用条件判断;}

\pt{中心极限定理在实际问题中的应用(如抽样估计);}

\pt{多个独立随机变量和的概率近似计算;}

\pt{中心极限定理与大数定律的综合应用}

