\chapter{参数估计}

\section{考点说明}
\begin{itemize}
\item 置信区间与置信水平;
\item 点估计的概念;
\item 最大似然估计法;
\item 单侧置信区间的求解
\item 非正态总体参数的置信区间(大样本情形);
\item 矩估计法;
\item 正态总体参数的置信区间(均值、方差、均值差、方差比);
\item 区间估计的概念;
\item 估计量的评选标准(无偏性、有效性、一致性)
\end{itemize}

\section{第一节 点估计}

\pt{矩估计法求解参数估计量;}

\pt{最大似然估计法求解参数估计量;}

\pt{估计量无偏性、有效性的判定;}

\pt{估计量一致性的判断}

\section{第二节 区间估计}

\pt{正态总体均值的置信区间求解;}

\pt{正态总体方差的置信区间求解;}

\pt{正态总体均值差、方差比的置信区间求解;}

\pt{置信水平与置信区间长度的关系判断;}

\pt{单侧置信区间的求解与应用}

