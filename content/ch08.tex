\chapter{行列式}

\section{考点说明}
\begin{itemize}
\item 克莱姆(Cramer)法则;
\item 行列式按行(列)展开定理
\item 行列式的性质;
\item 行列式的概念;
\item 伴随矩阵的性质
\item 矩阵可逆的充要条件(行列式不为零);
\end{itemize}

\section{第一节 行列式的概念与性质}

\pt{行列式的计算;}

\pt{利用行列式性质简化计算;}

\pt{高阶行列式的计算;}

\pt{行列式按行(列)展开定理的应用;}

\pt{爪型行列式的计算;}

\pt{范德蒙行列式的应用;}

\pt{含参数行列式的计算与讨论;}

\pt{分块矩阵行列式的计算;}

\pt{利用行列式证明等式或不等式}

\section{第二节 行列式的应用}

\pt{克莱姆法则求解线性方程组;}

\pt{利用行列式判断矩阵可逆性;}

\pt{伴随矩阵相关计算;}

\pt{克莱姆法则在含参数方程组中的应用;}

\pt{伴随矩阵与行列式的关系应用;}

\pt{利用行列式判断矩阵的秩;}

\pt{行列式在矩阵可逆性证明中的应用}

