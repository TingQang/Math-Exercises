% 定义自定义环境

% 考点环境
\newenvironment{exam-point}[2]{
  \par\vspace{2ex}
  \noindent\textbf{\textcolor{kaoyan-blue}{考点:#1}}\quad
  \label{#2}
}{\par\vspace{1ex}}

% 题型环境
\newenvironment{question-type}[2]{
  \subsection{#1} % 题型作为subsection
  \label{#2}
  \vspace{1ex}
}{\vspace{1ex}}

% 章节答案环境
\newenvironment{chapter-answers}{
  \section{本章答案与解析}
  \begin{center}
    \textbf{\large 答案与解析}
  \end{center}
  \vspace{1em}
}{}

% 题型答案环境
\newenvironment{type-answers}[1]{
  \subsection{#1 答案}
  \begin{enumerate}[label={【\arabic*】}]
}{\end{enumerate}}

% 答案项环境
\newenvironment{answer-item}[1]{
  \item
  \begin{tcolorbox}[colframe=answercolor, colback=answercolor!10, breakable]
    \textbf{\ref{#1}}:
}{\end{tcolorbox}}

% 为Exercise和Answer环境添加双向超链接支持
\makeatletter
% 简化ExerciseHeader,只添加中文标题和指向答案的链接
\renewcommand{\ExerciseHeader}{\medskip\centerline{\textbf{\Large 题型\arabic{Exercise}:\ExerciseTitle}(\refAnswer{\ExerciseLabel})}\vspace{1ex}}

% 简化AnswerHeader,使用中文标题,让宏包自动处理链接
\renewcommand{\AnswerHeader}{\medskip\centerline{\textbf{\large 答案与解析}}\smallskip}
\makeatother
