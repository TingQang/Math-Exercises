% ==============================================
% 自定义命令配置
% 文件:config/commands.tex
% 作用:集中管理所有自定义命令
% ==============================================

% 1. 数学符号快捷命令
\newcommand{\R}{\mathbb{R}}          % 实数集
\newcommand{\N}{\mathbb{N}}          % 自然数集
\newcommand{\Z}{\mathbb{Z}}          % 整数集
\newcommand{\Q}{\mathbb{Q}}          % 有理数集
\newcommand{\C}{\mathbb{C}}          % 复数集
\renewcommand{\P}{\mathbb{P}}        % 概率
\newcommand{\E}{\mathbb{E}}          % 期望

% 2. 微分与导数
\newcommand{\ddme}{\mathrm{d}}         % 微分符号
\newcommand{\pd}{\partial}           % 偏导符号
\newcommand{\pder}[2]{\frac{\partial #1}{\partial #2}}  % 偏导数
\newcommand{\gradme}{\nabla}           % 梯度
\newcommand{\diver}{\operatorname{div}} % 散度

% 3. 极限与积分
% 极限符号,行内公式中使用\limits确保下标在正下方
\newcommand{\Llim}{\lim\limits} % 带\limits的极限符号,用于行内公式
\newcommand{\Lint}{\int\limits} % 带\limits的积分符号,用于行内公式
\newcommand{\Liint}{\iint\limits} % 带\limits的二重积分符号
\newcommand{\Liiint}{\iiint\limits} % 带\limits的三重积分符号
\newcommand{\Loint}{\oint\limits} % 带\limits的曲线积分符号


% 恢复常用的极限快捷命令
\newcommand{\limn}{\Llim_{n\to\infty}}      % n→∞
\newcommand{\limx}{\Llim_{x\to\infty}}      % x→∞
\newcommand{\limxa}{\Llim_{x\to a}}         % x→a(通用)
\newcommand{\limxzero}{\Llim_{x\to 0}}         % x→0(常用)
\newcommand{\limxone}{\Llim_{x\to 1}}         % x→1(常用)
\newcommand{\limxinf}{\Llim_{x\to +\infty}} % x→+∞
\newcommand{\limxminf}{\Llim_{x\to -\infty}} % x→-∞

% 4. 矩阵与向量
\newcommand{\mat}[1]{\begin{pmatrix} #1 \end{pmatrix}}      % 圆括号矩阵
\newcommand{\bmat}[1]{\begin{bmatrix} #1 \end{bmatrix}}     % 方括号矩阵
\newcommand{\absme}[1]{\left| #1 \right|}                     % 绝对值
\newcommand{\normme}[1]{\left\| #1 \right\|}                  % 范数

% 5. 概率与统计
\newcommand{\Var}{\operatorname{Var}}    % 方差
\newcommand{\Cov}{\operatorname{Cov}}    % 协方差

% 6. 快捷输入
\newcommand{\dx}{\,\mathrm{d}x}          % 积分微元
\newcommand{\dy}{\,\mathrm{d}y}
\newcommand{\dz}{\,\mathrm{d}z}          % 积分微元z
\newcommand{\dt}{\,\mathrm{d}t}
\newcommand{\const}{\text{const}}        % 常数

% 7. 考点与习题专用命令
\newcommand{\keypoint}[1]{\textbf{\textcolor{kaoyan-red}{【考点】#1}}}
\newcommand{\important}[1]{\textbf{\textcolor{kaoyan-blue}{【重点】#1}}}

% 难度标注系统 - 基于TikZ的彩色方块标注
\newcommand{\difficulty}[1]{%
    【
    \ifnum#1=1
        \tikz{\node[fill=difficulty-1, draw=difficulty-1, minimum size=0.3em, inner sep=0, outer sep=0] {};}
        \tikz{\node[draw=difficulty-1, minimum size=0.3em, inner sep=0, outer sep=0] {};}
        \tikz{\node[draw=difficulty-1, minimum size=0.3em, inner sep=0, outer sep=0] {};}
        \tikz{\node[draw=difficulty-1, minimum size=0.3em, inner sep=0, outer sep=0] {};}
        \tikz{\node[draw=difficulty-1, minimum size=0.3em, inner sep=0, outer sep=0] {};}
    \else\ifnum#1=2
        \tikz{\node[fill=difficulty-2, draw=difficulty-2, minimum size=0.3em, inner sep=0, outer sep=0] {};}
        \tikz{\node[fill=difficulty-2, draw=difficulty-2, minimum size=0.3em, inner sep=0, outer sep=0] {};}
        \tikz{\node[draw=difficulty-2, minimum size=0.3em, inner sep=0, outer sep=0] {};}
        \tikz{\node[draw=difficulty-2, minimum size=0.3em, inner sep=0, outer sep=0] {};}
        \tikz{\node[draw=difficulty-2, minimum size=0.3em, inner sep=0, outer sep=0] {};}
    \else\ifnum#1=3
        \tikz{\node[fill=difficulty-3, draw=difficulty-3, minimum size=0.3em, inner sep=0, outer sep=0] {};}
        \tikz{\node[fill=difficulty-3, draw=difficulty-3, minimum size=0.3em, inner sep=0, outer sep=0] {};}
        \tikz{\node[fill=difficulty-3, draw=difficulty-3, minimum size=0.3em, inner sep=0, outer sep=0] {};}
        \tikz{\node[draw=difficulty-3, minimum size=0.3em, inner sep=0, outer sep=0] {};}
        \tikz{\node[draw=difficulty-3, minimum size=0.3em, inner sep=0, outer sep=0] {};}
    \else\ifnum#1=4
        \tikz{\node[fill=difficulty-4, draw=difficulty-4, minimum size=0.3em, inner sep=0, outer sep=0] {};}
        \tikz{\node[fill=difficulty-4, draw=difficulty-4, minimum size=0.3em, inner sep=0, outer sep=0] {};}
        \tikz{\node[fill=difficulty-4, draw=difficulty-4, minimum size=0.3em, inner sep=0, outer sep=0] {};}
        \tikz{\node[fill=difficulty-4, draw=difficulty-4, minimum size=0.3em, inner sep=0, outer sep=0] {};}
        \tikz{\node[draw=difficulty-4, minimum size=0.3em, inner sep=0, outer sep=0] {};}
    \else
        \tikz{\node[fill=difficulty-5, draw=difficulty-5, minimum size=0.3em, inner sep=0, outer sep=0] {};}
        \tikz{\node[fill=difficulty-5, draw=difficulty-5, minimum size=0.3em, inner sep=0, outer sep=0] {};}
        \tikz{\node[fill=difficulty-5, draw=difficulty-5, minimum size=0.3em, inner sep=0, outer sep=0] {};}
        \tikz{\node[fill=difficulty-5, draw=difficulty-5, minimum size=0.3em, inner sep=0, outer sep=0] {};}
        \tikz{\node[fill=difficulty-5, draw=difficulty-5, minimum size=0.3em, inner sep=0, outer sep=0] {};}
    \fi\fi\fi\fi
    】
}
\newcommand{\exercisestar}[1]{\difficulty{#1}} % 简写版本
\newcommand{\formula}[1]{\textcolor{kaoyan-blue}{\[#1\]}}

% 8. 习题扩展命令
\newcommand{\exercisesource}[1]{\textbf{\textcolor{gray}{【来源】#1}}} % 习题来源
\newcommand{\matchexampoint}[1]{\textbf{\textcolor{kaoyan-blue}{【匹配考点】#1}}} % 考点匹配
\newcommand{\exercisetip}[1]{\textbf{\textcolor{kaoyan-orange}{【提示】#1}}} % 答题提示
\newcommand{\multiplesolution}[1]{\textbf{\textcolor{kaoyan-orange}{【#1】}}} % 一题多解标注

% 9. 公式格式化命令
\newcommand{\longmath}[1]{\displaystyle #1} % 长行内公式使用displaystyle
\newcommand{\splitmath}[1]{\begin{split} #1 \end{split}} % 用于分割长公式
\newcommand{\multmath}[1]{\begin{multline} #1 \end{multline}} % 用于多行公式,自动换行

% 10. 题型样式命令
\newcounter{problemtype}[subsection]%题型计数器,随subsection重置
\renewcommand{\theproblemtype}{题型\arabic{problemtype}}
\newcommand{\pt}[1]{%
    \par\vspace{1em}% 题型上方间距
    \stepcounter{problemtype}% 增加题型计数器
    % 核心:居中 + 字体 + 字号设置
    \begin{center} % 居中环境(比 \centering 更稳定)
        % 字体+字号:推荐组合(兼顾美观和可读性)
        \zihao{4} % 中文四号字(≈14pt,题型标题黄金字号)
        \bfseries % 加粗
        \kaishu % 中文用楷体(典雅易读),也可换 \heiti/\songti
        \textbf{【\theproblemtype:#1】}% 输出题型编号和参数
    \end{center}
    \par\vspace{0.8em}% 题型下方间距
}


