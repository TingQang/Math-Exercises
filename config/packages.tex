% ==============================================
% 宏包加载配置
% 文件:config/packages.tex
% 作用:集中管理所有宏包加载
% ==============================================

% 第一部分:编码与字体
\usepackage[UTF8]{ctex}                      % 中文支持
\usepackage{anyfontsize}                     % 任意字体大小

% 第二部分:页面布局
\usepackage[
    a4paper,
    left=3cm,               % 左侧边距
    right=2.5cm,
    top=2.5cm,
    bottom=2.5cm,
    headheight=15pt,
    headsep=1cm,            % 页眉与正文间距
    footskip=1cm,           % 页脚与正文间距
    marginparwidth=0cm,     % 移除边注
    marginparsep=0cm
]{geometry}

% 第三部分:数学宏包
\usepackage{amsmath, amssymb, amsfonts}  % AMS数学基础(移除amsthm)
\usepackage{mathtools}                           % 增强数学工具
\usepackage{bm}                                  % 粗体数学符号
\usepackage{cases}                               % 多行公式
\usepackage{siunitx}                             % 单位与数字
\usepackage{cancel}                              % 取消线
\usepackage{esint}                               % 提供标准的三重积分符号
\usepackage{unicode-math}                        % 数学字体支持,解决字符缺失问题
\setmathfont{Latin Modern Math}                 % 设置数学字体

% 第四部分:图形与颜色
\usepackage{graphicx}                    % 图片插入
\usepackage{caption}                     % 支持\captionof命令
\usepackage{tcolorbox}                   % 彩色框体与定理环境(可折行、自定义样式)
\tcbuselibrary{most, breakable, skins}   % 加载tcolorbox扩展库(增强功能)
\usepackage{tikz}                        % 矢量图形绘制
\usetikzlibrary{shapes.geometric}
\usetikzlibrary{svg.path}
\usepackage[dvipsnames, svgnames, x11names]{xcolor}  % 颜色,添加选项以确保所有颜色命令都能正确定义
\usepackage{pgffor}                      % 循环

% 第五部分:排版与工具
\usepackage{etoolbox}                    % 工具宏包
\usepackage{enumitem}                    % 列表控制
% 修复section标题对齐问题并调整格式
\usepackage{titlesec}
\titleformat{\section}[hang]{\bfseries\large}{\thesection}{1em}{}

% 设置subsection标题为1,2,3形式并居中对齐
\titleformat{\subsection}[block]{\bfseries\centering}{\arabic{subsection}}{1em}{}
% 加载xsim宏包用于习题与答案管理
\usepackage{xsim}



\usepackage{booktabs}                    % 专业表格
\usepackage{multirow, multicol}          % 表格与多列
\usepackage{array}                       % 数组扩展
\usepackage{makecell}                    % 表格单元格
\usepackage{longtable}                   % 长表格

% 第六部分:页面样式
\usepackage{fancyhdr}                    % 页眉页脚
\usepackage{titletoc}                    % 目录格式
\usepackage{appendix}                    % 附录

% 第七部分:参考文献
\usepackage[backend=biber, style=gb7714-2015, gbpub=false]{biblatex}

% 第八部分:引用与链接(放在最后加载)
\usepackage{hyperref}                    % 超链接
\usepackage{cleveref}                    % 智能引用
