% ==============================================
% 考研数学笔记 - 完整宏包与自定义配置
% 文件:preamble.tex
% 作用:集中管理所有宏包、环境、自定义命令
% ==============================================

% ==================== 第一部分:编码与字体 ====================
\usepackage[UTF8]{ctex}                      % 中文支持
\usepackage{anyfontsize}                     % 任意字体大小

% ==================== 第二部分:页面布局 ====================
\usepackage[
    a4paper,
    left=3cm,               % 左侧边距(保留原有3cm设置)
    right=2.5cm,
    top=2.5cm,
    bottom=2.5cm,
    headheight=15pt,
    headsep=1cm,            % 页眉与正文间距
    footskip=1cm,           % 页脚与正文间距
    marginparwidth=0cm,     % 移除边注,使页面更干净
    marginparsep=0cm
]{geometry}

% ==================== 第三部分:数学宏包 ====================
\usepackage{amsmath, amsthm, amssymb, amsfonts}  % AMS数学基础
\usepackage{mathtools}                           % 增强数学工具
\usepackage{bm}                                  % 粗体数学符号
\usepackage{cases}                               % 多行公式
\usepackage{siunitx}                             % 单位与数字
\usepackage{cancel}                              % 取消线(原有需求)

% ==================== 第四部分:图形与颜色 ====================
\usepackage{graphicx}                    % 图片插入
\usepackage{tcolorbox}                   % 彩色框(含定理环境)
\tcbuselibrary{most, breakable, skins}   % 加载tcolorbox库
\usetikzlibrary{shapes.geometric}
\usetikzlibrary{svg.path}
\usepackage{xcolor}                      % 颜色
\usepackage{pgffor}                      % 循环

% 定义考研专用颜色
\definecolor{kaoyan-blue}{RGB}{0, 102, 204}    % 主色调蓝色
\definecolor{kaoyan-red}{RGB}{204, 51, 0}      % 重点红色
\definecolor{kaoyan-green}{RGB}{0, 153, 0}     % 正确/答案绿色
\definecolor{kaoyan-orange}{RGB}{255, 153, 0}  % 警告/注意橙色
\definecolor{kaoyan-gray}{RGB}{245, 245, 245}  % 背景灰色

% 习题难度分级颜色
\definecolor{difficulty-1}{RGB}{0, 153, 0}     % 难度1:绿色
\definecolor{difficulty-2}{RGB}{0, 102, 204}   % 难度2:蓝色
\definecolor{difficulty-3}{RGB}{255, 153, 0}   % 难度3:橙色
\definecolor{difficulty-4}{RGB}{204, 51, 0}    % 难度4:红色
\definecolor{difficulty-5}{RGB}{102, 0, 153}   % 难度5:紫色

% 星形颜色
\definecolor{star-full}{RGB}{255,215,0}  % 实心星:金色

% ==================== 第五部分:排版与工具 ====================
\usepackage{enumitem}                    % 列表控制
\usepackage{booktabs}                    % 专业表格
\usepackage{multirow, multicol}          % 表格与多列
\usepackage{array}                       % 数组扩展
\usepackage{makecell}                    % 表格单元格
\usepackage{longtable}                   % 长表格

% 段落格式配置
\setlength{\parindent}{2em}      % 首行缩进2字符
\setlength{\parskip}{0.5em}      % 段落间距0.5em

% ==================== 第六部分:页面样式 ====================
\usepackage{fancyhdr}                    % 页眉页脚
\usepackage{titlesec}                    % 标题格式
\usepackage{titletoc}                    % 目录格式
\usepackage{appendix}                    % 附录

% ==================== 第七部分:引用与链接 ====================
\usepackage{hyperref}                    % 超链接
\usepackage{cleveref}                    % 智能引用

% ==================== 第八部分:参考文献 ====================
\usepackage[backend=biber, style=gb7714-2015, gbpub=false]{biblatex}
% ==================== 第九部分:自定义环境 ====================

% 1. 定理类环境
\newtcbtheorem[number within=chapter]{definition}{定义}{%
    colframe=kaoyan-blue, 
    colback=kaoyan-gray, 
    colbacktitle=kaoyan-blue!10, 
    coltitle=kaoyan-blue, 
    fonttitle=\bfseries, 
    boxed title style={size=small},
    breakable
}{def}

\newtcbtheorem[number within=chapter]{theorem}{定理}{%
    colframe=kaoyan-red, 
    colback=kaoyan-gray!30, 
    colbacktitle=kaoyan-red!10,
    coltitle=kaoyan-red, 
    fonttitle=\bfseries, 
    boxed title style={size=small},
    breakable
}{thm}

\newtcbtheorem[number within=chapter]{lemma}{引理}{%
    colframe=kaoyan-green, 
    colback=kaoyan-gray!30, 
    colbacktitle=kaoyan-green!10,
    coltitle=kaoyan-green, 
    fonttitle=\bfseries, 
    boxed title style={size=small},
    breakable
}{lem}

\newtcbtheorem[number within=chapter]{corollary}{推论}{%
    colframe=kaoyan-orange, 
    colback=kaoyan-gray!30, 
    colbacktitle=kaoyan-orange!10,
    coltitle=kaoyan-orange, 
    fonttitle=\bfseries, 
    boxed title style={size=small},
    breakable
}{cor}

\newtcbtheorem[number within=section]{remark}{注}{%
    colframe=gray, 
    colback=kaoyan-gray, 
    colbacktitle=gray!10,
    coltitle=gray, 
    fonttitle=\bfseries, 
    boxed title style={size=small},
    breakable
}{rem}

% 2. 考研习题专用环境
\newtcbtheorem[number within=chapter]{exam-point}{考点}{%
    colframe=kaoyan-blue, 
    colback=kaoyan-gray, 
    colbacktitle=kaoyan-blue!10, 
    coltitle=kaoyan-blue, 
    fonttitle=\bfseries, 
    boxed title style={size=small},
    breakable,
    attach boxed title to top left={xshift=1em, yshift=-0.5em}
}{ep}

\newtcolorbox{question-type}[1]{
    colframe=kaoyan-blue!70!black,
    colback=kaoyan-blue!5,
    title=题型:#1,
    fonttitle=\bfseries,
    breakable,
    attach boxed title to top left={xshift=1em, yshift=-0.3em}
}

\newtcolorbox{exercise-item}[2][]{
    colframe=gray!50!black,
    colback=white,
    title=习题 #2,
    fonttitle=\bfseries,
    breakable,
    #1,
    attach boxed title to top left={xshift=1em, yshift=-0.3em}
}

\newtcolorbox{error-analysis}{
    colframe=difficulty-4,
    colback=difficulty-4!5,
    title=常见错因,
    fonttitle=\bfseries,
    breakable,
    attach boxed title to top left={xshift=1em, yshift=-0.3em}
}

\newtcolorbox{exercise-solution}{
    colframe=kaoyan-green!50!black,
    colback=kaoyan-green!3,
    title=解析,
    fonttitle=\bfseries,
    breakable,
    attach boxed title to top left={xshift=1em, yshift=-0.3em}
}

\newtcolorbox{example}{
    colframe=kaoyan-green!70!black,
    colback=kaoyan-green!5,
    title=例题,
    fonttitle=\bfseries,
    breakable
}

\newtcolorbox{exercise}{
    colframe=kaoyan-blue!70!black,
    colback=kaoyan-blue!5,
    title=习题,
    fonttitle=\bfseries,
    breakable
}

\newtcolorbox{solution}{
    colframe=kaoyan-green!50!black,
    colback=kaoyan-green!3,
    title=解,
    fonttitle=\bfseries,
    breakable
}

\newtcolorbox{proofbox}{
    colframe=kaoyan-red!50!black,
    colback=kaoyan-red!3,
    title=证明,
    fonttitle=\bfseries,
    breakable
}

\newtcolorbox{summary}{
    colframe=kaoyan-orange,
    colback=kaoyan-orange!5,
    title=本章总结,
    fonttitle=\bfseries,
    breakable
}

\newtcolorbox{attention}{
    colframe=kaoyan-red,
    colback=kaoyan-red!5,
    title=注意,
    fonttitle=\bfseries,
    breakable
}

% ==================== 第十部分:自定义命令(考研数学专用) ====================

% 1. 数学符号快捷命令
\newcommand{\R}{\mathbb{R}}          % 实数集
\newcommand{\N}{\mathbb{N}}          % 自然数集
\newcommand{\Z}{\mathbb{Z}}          % 整数集
\newcommand{\Q}{\mathbb{Q}}          % 有理数集
\newcommand{\C}{\mathbb{C}}          % 复数集
\renewcommand{\P}{\mathbb{P}}        % 概率
\newcommand{\E}{\mathbb{E}}          % 期望

% 2. 微分与导数
\newcommand{\ddme}{\mathrm{d}}         % 微分符号
\newcommand{\pd}{\partial}           % 偏导符号
\newcommand{\pder}[2]{\frac{\partial #1}{\partial #2}}  % 偏导数
\newcommand{\gradme}{\nabla}           % 梯度
\newcommand{\diver}{\operatorname{div}} % 散度

% 3. 极限与积分
\newcommand{\limn}{\lim\limits_{n\to\infty}}
\newcommand{\limx}{\lim\limits_{x\to\infty}}
\newcommand{\limxa}{\lim\limits_{x\to a}}
\newcommand{\intab}{\int_{a}^{b}}
\newcommand{\iintD}{\iint\limits_{D}}
\newcommand{\ointC}{\oint\limits_{C}}

% 4. 矩阵与向量
\newcommand{\mat}[1]{\begin{pmatrix} #1 \end{pmatrix}}      % 圆括号矩阵
\newcommand{\bmat}[1]{\begin{bmatrix} #1 \end{bmatrix}}     % 方括号矩阵
\newcommand{\absme}[1]{\left| #1 \right|}                     % 绝对值
\newcommand{\normme}[1]{\left\| #1 \right\|}                  % 范数

% 5. 概率与统计
\newcommand{\Var}{\operatorname{Var}}    % 方差
\newcommand{\Cov}{\operatorname{Cov}}    % 协方差

% 6. 快捷输入
\newcommand{\dx}{\,\mathrm{d}x}          % 积分微元
\newcommand{\dy}{\,\mathrm{d}y}
\newcommand{\dt}{\,\mathrm{d}t}
\newcommand{\const}{\text{const}}        % 常数

% 7. 考点与习题专用命令(整合原有功能)
\newcommand{\keypoint}[1]{\textbf{\textcolor{kaoyan-red}{【考点】#1}}}
\newcommand{\important}[1]{\textbf{\textcolor{kaoyan-blue}{【重点】#1}}}

% 星级难度(整数级别 1-5)- 美化版
\newcommand{\fullstar}{\tikz{\node[star, star points=5, star point ratio=2.25, fill=star-full, draw=none, scale=0.25] {};} }
\newcommand{\difficulty}[1]{%
    \textcolor{difficulty-#1}{%
        \ifnum#1=1 \fullstar
        \else\ifnum#1=2 \fullstar\fullstar
        \else\ifnum#1=3 \fullstar\fullstar\fullstar
        \else\ifnum#1=4 \fullstar\fullstar\fullstar\fullstar
        \else\ifnum#1=5 \fullstar\fullstar\fullstar\fullstar\fullstar
        \fi\fi\fi\fi\fi
    }%
}
\newcommand{\exercisestar}[1]{\difficulty{#1}} % 简写版本
\newcommand{\formula}[1]{\textcolor{kaoyan-blue}{\[\#1\]}}

% 8. 习题扩展命令
\newcommand{\exercisesource}[1]{\textbf{\textcolor{gray}{【来源】#1}}} % 习题来源
\newcommand{\matchexampoint}[1]{\textbf{\textcolor{kaoyan-blue}{【匹配考点】#1}}} % 考点匹配
\newcommand{\exercisetip}[1]{\textbf{\textcolor{kaoyan-orange}{【提示】#1}}} % 答题提示
\newcommand{\multiplesolution}[1]{\textbf{\textcolor{kaoyan-orange}{【#1】}}} % 一题多解标注

% ==================== 第十一部分:页面样式设置 ====================

% 页眉页脚设置
\pagestyle{fancy}
\fancyhf{}
\fancyhead[L]{\leftmark}
\fancyhead[R]{\rightmark}
\fancyfoot[C]{\thepage}
\renewcommand{\headrulewidth}{0.4pt}
\renewcommand{\footrulewidth}{0pt}

% 章节格式设置(整合原有样式)
\titleformat{\chapter}[display]
    {\normalfont\Huge\bfseries\color{kaoyan-blue}}
    {\chaptertitlename\ \thechapter}{20pt}{\centering}
\titlespacing*{\chapter}{0pt}{-30pt}{40pt}

% 节格式设置
\titleformat{\section}
    {\normalfont\Large\bfseries\color{kaoyan-blue}}
    {\thesection}{1em}{}
\titlespacing*{\section}{0pt}{3.5ex plus 1ex minus .2ex}{2.3ex plus .2ex}

% 小节格式设置(保留原有需求)
\titleformat{\subsection}
    {\normalfont\large\bfseries\color{kaoyan-blue!80!black}}
    {\thesubsection}{1em}{}
\titlespacing*{\subsection}{0pt}{3.2ex plus 1ex minus .2ex}{1.8ex plus .2ex}

% 超链接设置(整合原有配置)
\hypersetup{
    colorlinks=true,
    linkcolor=kaoyan-blue,
    citecolor=kaoyan-green,
    filecolor=magenta,
    urlcolor=kaoyan-orange,  % 保留原有橙色链接
    linktoc=all,             % 保留原有目录链接配置
    bookmarksopen=true,
    pdftitle={考研数学笔记},
    pdfauthor={你的名字},
    pdfsubject={考研数学复习资料}
}

% 列表格式设置(整合原有样式)
\setlist[enumerate]{
    topsep=0pt,
    itemsep=0pt,
    parsep=0pt,
    label=\textcolor{kaoyan-blue}{\arabic*.)},  % 保留原有蓝色编号
    leftmargin=2em
}

\setlist[itemize]{
    topsep=0pt,
    itemsep=0pt,
    parsep=0pt,
    label=\textcolor{kaoyan-blue}{$\bullet$},    % 保留原有蓝色项目符号
    leftmargin=2em
}

% 表格设置
\setlength{\heavyrulewidth}{0.12em}
\setlength{\lightrulewidth}{0.05em}
\renewcommand{\arraystretch}{1.2}

% ==================== 第十二部分:计数器设置 ====================

% 公式编号带章节号
\numberwithin{equation}{chapter}

% 图片表格编号带章节号
\renewcommand{\thefigure}{\thechapter-\arabic{figure}}
\renewcommand{\thetable}{\thechapter-\arabic{table}}

% 重置计数器
\AtBeginDocument{
    \setcounter{chapter}{0}
    \setcounter{section}{0}
}